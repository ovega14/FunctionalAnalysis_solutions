\documentclass{article}
\usepackage[utf8]{inputenc}
\usepackage[english]{babel}
\usepackage[]{amsthm} %lets us use \begin{proof}
\usepackage[]{amssymb} %gives us the character \varnothing
\usepackage[]{amsmath}
\usepackage[parfill]{parskip} %avoid indent when skipping lines
\usepackage[toc,page]{appendix}
\usepackage{mathtools}
\usepackage{hyperref}
\hypersetup{
	colorlinks=true,
	linkcolor=blue,
	filecolor=magenta,      
	urlcolor=cyan,
	pdftitle={18.102-ps6},
	pdfpagemode=FullScreen,
}
%\urlstyle{same}
\newcommand{\R}{\mathbb{R}} %the real numbers
\newcommand{\N}{\mathbb{N}} %the natural numbers
\newcommand{\M}{\mathcal{M}} %set of all Lebesgue-measurable sets
\newcommand{\Q}{\mathbb{Q}} % the rational numbers
\newcommand{\C}{\mathbb{C}} %the complex numbers

\title{18.102 Assignment 6}
\author{Octavio Vega}
\date\today

\begin{document}
\maketitle
	
\section*{Problem 1}
\subsection*{(a)}
\begin{proof}
	Let $\epsilon > 0$, and define $c_1 := a$ and $c_{n+1} := b$. 
	
	Since $\psi$ is a step function on $[a, b]$, $\exists c_1 \leq c_2 \leq \cdots \leq c_n \leq c_{n+1} \in [a, b]$ such that $\forall i = 1, \cdots, n$,
	\begin{equation}\label{psi_inverse}
		\psi^{-1}(\{a_i\}) = (c_i, c_{i+1}],
	\end{equation}
	where each $a_i$ is one of the finitely many values that $\psi$ takes on. 
	
	Choose $\delta > 0$ such that $\delta < \frac{\epsilon}{2n}$. Define $g: [a, b] \rightarrow \R$ via
	\begin{align}
		g(x) := \begin{cases}
			\frac{a_i + a_{i-1}}{2} + \left(\frac{a_i - a_{i-1}}{2\delta}\right)(x-c_i), \; &x \in (c_i - \delta, c_i + \delta) \\
			a_i, \; &x \in [c_i + \delta, c_{i+1}-\delta] \\
			-\frac{a_n}{2\delta}(x-b), \; &x \in (c_{n-1}-\delta, b],
		\end{cases}
	\end{align}
	where $a_0 := -a_1$. Then
	\begin{equation}
		g(a) = \frac{a_1 + a_0}{2} + \left(\frac{a_1 - a_0}{2\delta}\right)(c_1 - c_1) = \frac{a_1 - a_1}{2} = 0,
	\end{equation}
	and
	\begin{equation}
		g(b) = -\frac{a_n}{2\delta}(b-b) = 0,
	\end{equation}
	as desired. We also see that since $g$ is piecewise linear, it is continuous. 
	
	Now consider the difference $|\psi(x) - g(x)|$.
	
	\underline{Case 1}: $x \in [c_i + \delta, c_{i+1} - \delta]$. Then by \eqref{psi_inverse}, we have
	\begin{equation}
		|\psi(x) - g(x)| = |\psi\left((c_i, c_{i+1}-\delta)\right) - a_i| = |a_i - a_i| = 0.
	\end{equation}
	\underline{Case 2}: $x \in (c_i - \delta, c_i + \delta)$. Then
	\begin{align}
		|\psi(x) - g(x)| &= \left|\frac{a_i + a_{i-1}}{2} + \left(\frac{a_i - a_{i-1}}{2\delta}\right)(x-c_i) - \psi(x)\right| \\
		&< \left|\frac{a_i + a_{i-1}}{2} + \left(\frac{a_i - a_{i-1}}{2\delta}\right)\delta - \psi(x)\right| \\
		&= \left|\frac{a_i + a_{i-1}}{2} + \frac{a_i + a_{i-1}}{2} - \psi(x)\right| \\
		&= |a_i - \psi(x)| \\
		&= 0 \textrm{ or } |a_{i+1} - a_i|.
	\end{align}
	\underline{Case 3}: $x \in (c_n - \delta, b)$. Then
	\begin{align}
		|\psi(x) - g(x)| &= \left|-\frac{a_n}{2\delta}(x-b) - \psi(x)\right| \\
		&< \left|-\frac{a_n}{2\delta}\delta - \psi(x)\right| \\
		&= \left|-\frac{a_n}{2} - \psi(x)\right| \\
		&= \frac{3a_n}{2}.
	\end{align}
	So in all three cases, we have that either $|g(x) - \psi(x)| = 0$, or $|g(x) - \psi(x)| < \frac{3a}{2}$, or $|g(x) - \psi(x)|< |a_{i+1} - a_i|$.
	
	Define the set $E$ to be the collection of points in $[a, b]$ for which $|\psi(x) - g(x)| \neq 0$. Then
	\begin{equation}
		E := \bigcup_{k=1}^{n} (c_k - \delta, c_k + \delta).
	\end{equation}
	By definition, $\forall x \in E^c$, 
	\begin{equation}
		|\psi(x) - g(x)| = 0 < \epsilon.
	\end{equation}
	Since $E$ is a countable union of intervals, we have
	\begin{align}
		m(E) &= m\left[\bigcup_{k=1}^n (c_k - \delta, c_k + \delta)\right] \\
		&\leq \sum_{k=1}^n m\left[(c_k - \delta, c_k + \delta)\right] \\
		&= \sum_{k=1}^n \ell (c_k - \delta, c_k + \delta) \\
		&= \sum_{k=1}^n 2\delta\\
		&= 2n\delta \\
		& < 2n \frac{\epsilon}{2n} \\
		&= \epsilon,
	\end{align}
	as desired.
\end{proof}

\subsection*{(b)}
\begin{proof}
	Suppose $E$ is Lebesgue measurable and $m(E) < \infty$. Then by Littlewood's first principle, $\forall \delta > 0$ $\exists$ a finite collection of open intervals $\{U_i\}_{i=1}^n$ such that
	\begin{equation}
		m\left(E \Delta \bigcup_{i=1}^n U_i\right) < \delta.
	\end{equation}
	Let $\epsilon > 0$. Since $\varphi$ is a simple function, then we can express it as
	\begin{equation}
		\varphi = \sum_{i=1}^n a_i \chi_{E_i},
	\end{equation}
	where we take $U_i$ such that the symmetric difference
	\begin{equation}
		m(U_i \backslash E_i) + m(E_i \backslash U_i) < \frac{\epsilon}{n}
	\end{equation}
	for each $i \in \{1, ..., n\}$. 
	
	Let $\psi = \sum_{i=1}^n a_i \chi_{U_i}$. Then $\forall x \in U \cap E = (U \Delta E)^c$,
	\begin{equation}
		|\varphi(x) - \psi(x)| = 0 < \epsilon.
	\end{equation}
	We have
	\begin{align}
		m(U \Delta E) &= \sum_{i=1}^n m(U_i \backslash E_i) + \sum_{i=1}^n m(E_i \backslash U_i) \\
		&< n \frac{\epsilon}{n} = \epsilon,
	\end{align}
	as desired.
\end{proof}
%%%%%%%%%%%%%%%%%%%%%%%%%%%%%%%%%%%%%%%%%%%%%%%%%%%%%%%%%%%%%%%%%%%%%%%%%%%%%%%%%%%%%%%%%%%%%%%%%%%%%%%%%%%%%%%%%%%%%%%%%%%%%%%%%%%%%%%%%%%%
\section*{Problem 2}
\begin{proof}
	Let $c_0=a$, $c_n=b$, and partition $[a, b]$ into disjoint intervals as follows:
	\begin{equation}
		[a, b] = [c_0, c_1) \cup [c_1, c_2) \cup ... \cup [c_{n-1}, c_n) \cup \{b\}.
	\end{equation}
	Letting $U_k = [c_{k-1}, c_k)$ for each $k \in \{1, ... n\}$, we have
	\begin{equation}
		\bigcup_{k=1}^n U_k = [a, b).
	\end{equation}
	On each interval $U_k$, extract a closed set $F_k$ such that $f$ is continuous on $F_k$ and such that
	\begin{equation}
		m\left(\bigcup_{k=1}^n F_k^c\right) < \epsilon.
	\end{equation}
	Then for each $k \in \{1, ..., n\}$, $\exists F_k \subseteq U_k$ closed such that $m(E \backslash F_k) < \frac{\epsilon}{2^n}$ and such that $f\big|_{F_k}$ is continuous.
	
	Since $F_k$ is closed, then $F_k^c$ is open, so the union $\bigcup_{k=1}^n F_k^c$ is open as well. Then $E = \bigcup_{k=1}^n F_k^c$ can be expressed as a union of disjoint intervals:
	\begin{equation}
		E = \bigcup_{k=1}^n (a_k, b_k).
	\end{equation}
	Define $g: [a, b] \to \R$ via
	\begin{align}
		g(x) := \begin{cases}
			\frac{x - a_k}{b_k - a_k}f(b_k) + f(a_k), \quad &x \in E \\
			f(x), \quad &x \in E^c.
		\end{cases}
	\end{align}
	Then
	\begin{equation}
		\sup_{x \in [a, b]}|g(x)| \leq \sup_{x \in E}|g(x)| \leq |f(a_k)| + |f(b_k)|,
	\end{equation}
	and
	\begin{equation}
		\sup_{x \in [a, b]}|g(x)| \leq \sup_{x \in E^c}|g(x)| \leq \sup_{x \in E^c}|f(x)| \leq B.
	\end{equation}
	Also, 
	\begin{equation}
		m(E) = \sum_{k=1}^n m(F_k^c) < \sum_{k=1}^n \frac{\epsilon}{2^n} = \epsilon.
	\end{equation}
	Finally, $\forall x \in E^c$, $g(x) = f(x)$, so $|f(x) - g(x)| = 0 < \epsilon$.
	
	Therefore, the function $g$ meets the desired requirements.
\end{proof}
%%%%%%%%%%%%%%%%%%%%%%%%%%%%%%%%%%%%%%%%%%%%%%%%%%%%%%%%%%%%%%%%%%%%%%%%%%%%%%%%%%%%%%%%%%%%%%%%%%%%%%%%%%%%%%%%%%%%%%%%%%%%%%%%%%%%%%%%%%%%
\section*{Problem 3}
\subsection*{(a)}
\begin{proof}
	Since $f$ is Lebesgue integrable, then $f$ is measurable. For each $n \in \N$, let $E_n = f^{-1}([-n, n])$. Then $[-n, n] = [-\infty, n] \cap [-n, \infty]$ and since $f$ is measurable, then $f^{-1}([-\infty, n])$ and $f^{-1}([-n, \infty])$ are measurable. Thus \\$f^{-1}([-\infty, n] \cap [-n, \infty])$ is measurable. Hence $f^{-1}([-n, n])$ is measurable $\iff E_n$ is measurable.
	
	Define $h_n = f\chi_{E_n}$. Then
	\begin{align}
		0 &\leq \int_a^b |f(x) - h_n(x)| \mathrm{d}x\\
		&= \int_a^b |f(x) - f(x)\chi_{E_n}(x)| \mathrm{d}x \\
		&= \int_a^b |f(x)||1 - \chi_{E_n}(x)| \mathrm{d}x \\
		&= \int_a^b |f(x)|\chi_{E_n^c}(x) \mathrm{d}x \\
		&= \int_{[a, b]\cap E_n^c}|f(x)| \mathrm{d}x,
	\end{align}
	where we have used the simple fact that for a measurable set $E$, the characteristic function over the set's complement takes the form $\chi_{E^c} = 1 - \chi_{E}$.
	
	We have $[a, b] \cap E_n^c = [a, b] \cap f^{-1}([-n, n])^c$. But as $n \to \infty$, \\$f^{-1}([-n, n]) \to [a, b]$, so $f^{-1}([-n, n])^c \to [a, b]^c$. Hence, $[a, b] \cap E_n^c \to \varnothing$.
	
	Sending $n \to \infty$, we get
	\begin{align}
		0 &\leq \lim_{n \to \infty}\int_a^b |f(x) - h_n(x)| \mathrm{d}x \\
		&= \lim_{n \to \infty} \int_{[a, b]\cap E_n^c} |f(x)| \mathrm{d}x \\
		&= \int_{\varnothing}|f(x)| \mathrm{d}x \\
		&= 0.
	\end{align}
	Thus, $\lim_{n \to \infty}\int_a^b |f(x) - h_n(x)| \mathrm{d}x = 0$.
	
	Let $h = \lim_{n \to \infty}h_n = f \chi_{[a, b]}$. Then $h$ is a product of two measurable functions, so $h$ is measurable. Also, $h$ is the limit of a sequence of simple functions $\{h_n\}_n$ with $|h| \leq |h_n|$ $\forall n \in \N$. So $h$ is bounded.
	
	Therefore $\int_{a}^b |f(x) - h(x)|\mathrm{d}x < \epsilon$, as desired.
\end{proof}
	
\subsection*{(b)}
\begin{proof}
	By \textbf{(a)}, we know that $\exists$ a bounded measurable function $h: [a, b] \to \R$ such that
	\begin{equation}
		\int_a^b |f(x) - h(x)| \mathrm{d}x < \frac{\epsilon}{3}.
	\end{equation}
	Next, we approximate $h$ by a simple function $\varphi: [a, b] \to \R$ such that
	\begin{equation}
		\int_a^b |h(x) - \phi(x)| \mathrm{d}x < \frac{\epsilon}{3}.
	\end{equation}
	Finally, by problem \textbf{(1b)}, we can approximate $\varphi$ by a step function $\psi$ such that
	\begin{equation}\label{3b.i}
		|\varphi(x) - \psi(x)| < \frac{\epsilon}{3(b-a)}.
	\end{equation}
	Integrating \eqref{3b.i} over $[a, b]$, this is equivalent to
	\begin{equation}
		\int_a^b |\varphi(x) - \psi(x)| \mathrm{d}x < \frac{\epsilon}{3(b-a)}m([a, b]) = \frac{\epsilon}{3}.
	\end{equation}
	So, by the triangle inequality we have
	\begin{align}
		\int_a^b |f(x) - \psi(x)| \mathrm{d}x = \int_a^b |f(x) - h(x) + h(x) - \varphi(x) + \varphi(x) - \psi(x)| \mathrm{d}x \\
		\leq \int_a^b |f(x) - h(x)| \mathrm{d}x + \int_a^b |h(x) - \varphi(x)| \mathrm{d}x + \int_a^b |\varphi(x) - \psi(x)|\mathrm{d}x \\
		< \frac{\epsilon}{3} + \frac{\epsilon}{3} + \frac{\epsilon}{3} \\
		= \epsilon.
	\end{align}
	Thus, $\psi$ suffices.
\end{proof}

\subsection*{(c)}
\begin{proof}
	Let $E = \{x\in [a, b] \;|\; |f(x) - g(x)| \geq \frac{\epsilon}{b-a}\}$. 
	
	By what we showed in problem \textbf{2}, we know $\exists$ such a continuous function $g: [a, b] \to \R$ such that  $g(a) = g(b) = 0$ and 
	\begin{equation}
		m\left(E\right) = 0.
	\end{equation}
	Then we have
	\begin{align}
		\int_a^b |f(x) - g(x)| \mathrm{d}x &= \int_{E\cup E^c} |f(x) - g(x)| \mathrm{d}x \\
		&= \int_E |f(x) - g(x)| \mathrm{d}x + \int_{E^c} |f(x) - g(x)| \mathrm{d}x.
	\end{align}
	But $m(E) = 0$, so $m(E^c) = b-a$. Then
	\begin{align}
		\int_a^b |f(x) - g(x)| \mathrm{d}x &= \int_{E^c} |f(x) - g(x)| \mathrm{d}x \\
		&< \int_{E^c} \frac{\epsilon}{b-a} \mathrm{d}x \\
		&= (b-a) \frac{\epsilon}{b-a} \\
		&= \epsilon.
	\end{align}
	Hence, $g$ satisfies the desired constraints.
\end{proof}
%%%%%%%%%%%%%%%%%%%%%%%%%%%%%%%%%%%%%%%%%%%%%%%%%%%%%%%%%%%%%%%%%%%%%%%%%%%%%%%%%%%%%%%%%%%%%%%%%%%%%%%%%%%%%%%%%%%%%%%%%%%%%%%%%%%%%%%%%%%%
\section*{Problem 4}
\begin{proof}
	First, we prove the Riemann-Lebesgue Lemma for step functions.
	
	Let $\psi: [-\pi, \pi] \to \C$ be a step function. Then given a partition \\$-\pi = x_0 < x_1 < ... < x_m = \pi$, we may write
	\begin{equation}
		\psi = \sum_{k=1}^m c_k \chi_{[x_{k-1}, x_k)}.
	\end{equation}
	Consider each of the integrals
	\begin{equation}
		I_1 = \int_{-\pi}^{\pi} \psi \cos{(nx)}\mathrm{d}x
	\end{equation}
	and
	\begin{equation}
		I_2 = \int_{-\pi}^{\pi}\psi \sin{(nx)} \mathrm{d}x.
	\end{equation}
	Computing, we get
	\begin{align}
		I_1 &= \sum_{k=1}^m c_k \int_{-\pi}^{\pi}\chi_{[x_{k-1}, x_k)}\cos{(nx)}\mathrm{d}x \\
		&= \sum_{k=1}^m c_k \int_{x_{k-1}}^{x_k}\cos{(nx)} \mathrm{d}x \\
		&= \sum_{k=1}^m c_k \frac{1}{n} \left[\sin{(x_k)} - \sin{(x_{k-1})}\right].
	\end{align}
	Similarly, we find
	\begin{equation}
		I_2 = -\sum_{k=1}^m c_k \frac{1}{n}\left[\cos{(x_k)} - \cos{(x_{k-1})}\right].
	\end{equation}
	Combining these results, we have
	\begin{align}
		\hat{\psi}(n) &= \frac{1}{2\pi}(I_1 - iI_2) \\
		&= \frac{1}{2\pi n}\sum_{k=1}^m c_k \left[\sin{(x_k)} - \sin{(x_{k-1})} + i \left(\cos{(x_k)} - \cos{(x_{k-1})}\right)\right].
	\end{align}
	Taking the magnitude, we are left with
	\begin{align}
		|\hat{\psi}(n)| &\leq \sum_{k=1}^m \frac{|c_k|}{n}\left|\sin{(x_k)} - \sin{(x_{k-1})} + i \left(\cos{(x_k)} - \cos{(x_{k-1})}\right)\right| \\
		&\leq 4 \sum_{k=1}^m \frac{|c_k|}{n} \xrightarrow{n \to \infty} 0.
	\end{align}
	Thus, $\lim_{|n| \to \infty}|\hat{\psi}(n)| = 0$.
	
	By problem \textbf{3a}, this step function $\psi$ exists for a Lebesgue integrable function $f: [-\pi, \pi] \to \C$, i.e. for $\epsilon > 0$, $\exists \psi$ such that
	\begin{equation}
		\int_{-\pi}^{\pi} \left(f(x) - \psi(x)\right) e^{-inx} \mathrm{d}x < \epsilon.
	\end{equation}
	So we have
	\begin{align}
		0 &\leq \left|\int_{-\pi}^{\pi} f(x)e^{-inx}\mathrm{d}x - \int_{-\pi}^{\pi}\psi(x)e^{-inx}\mathrm{d}x\right| \\
		&\leq \int_{-\pi}^{\pi} \left|(f(x) - \psi(x))e^{-inx}\right| \mathrm{d}x \\
		&< \epsilon.
	\end{align}
	Therefore, $\lim_{|n| \to \infty} |\hat{f}(n)|$ = 0.
\end{proof}

\end{document}
