\documentclass{article}
\usepackage[utf8]{inputenc}
\usepackage[english]{babel}
\usepackage[]{amsthm} %lets us use \begin{proof}
\usepackage[]{amssymb} %gives us the character \varnothing
\usepackage[]{amsmath}
\usepackage[parfill]{parskip} %avoid indent when skipping lines
\usepackage[toc,page]{appendix}
\usepackage{mathtools}
\usepackage{hyperref}
\hypersetup{
	colorlinks=true,
	linkcolor=blue,
	filecolor=magenta,      
	urlcolor=cyan,
	pdftitle={18.102-ps6},
	pdfpagemode=FullScreen,
}
%\urlstyle{same}
\newcommand{\R}{\mathbb{R}} %the real numbers
\newcommand{\N}{\mathbb{N}} %the natural numbers
\newcommand{\M}{\mathcal{M}} %set of all Lebesgue-measurable sets
\newcommand{\Q}{\mathbb{Q}} % the rational numbers

\title{18.102 Assignment 6}
\author{Octavio Vega}
\date\today

\begin{document}
\maketitle
	
\section*{Problem 1}
\subsection*{(a)}
\begin{proof}
	Let $\epsilon > 0$, and define $c_1 := a$ and $c_{n+1} := b$. 
	
	Since $\psi$ is a step function on $[a, b]$, $\exists c_1 \leq c_2 \leq \cdots \leq c_n \leq c_{n+1} \in [a, b]$ such that $\forall i = 1, \cdots, n$,
	\begin{equation}\label{psi_inverse}
		\psi^{-1}(\{a_i\}) = (c_i, c_{i+1}],
	\end{equation}
	where each $a_i$ is one of the finitely many values that $\psi$ takes on. 
	
	Choose $\delta > 0$ such that $\delta < \frac{\epsilon}{2n}$. Define $g: [a, b] \rightarrow \R$ via
	\begin{align}
		g(x) := \begin{cases}
			\frac{a_i + a_{i-1}}{2} + \left(\frac{a_i - a_{i-1}}{2\delta}\right)(x-c_i), \; &x \in (c_i - \delta, c_i + \delta) \\
			a_i, \; &x \in [c_i + \delta, c_{i+1}-\delta] \\
			-\frac{a_n}{2\delta}(x-b), \; &x \in (c_{n-1}-\delta, b],
		\end{cases}
	\end{align}
	where $a_0 := -a_1$. Then
	\begin{equation}
		g(a) = \frac{a_1 + a_0}{2} + \left(\frac{a_1 - a_0}{2\delta}\right)(c_1 - c_1) = \frac{a_1 - a_1}{2} = 0,
	\end{equation}
	and
	\begin{equation}
		g(b) = -\frac{a_n}{2\delta}(b-b) = 0,
	\end{equation}
	as desired. We also see that since $g$ is piecewise linear, it is continuous. 
	
	Now consider the difference $|\psi(x) - g(x)|$.
	
	\underline{Case 1}: $x \in [c_i + \delta, c_{i+1} - \delta]$. Then by \eqref{psi_inverse}, we have
	\begin{equation}
		|\psi(x) - g(x)| = |\psi\left((c_i, c_{i+1}-\delta)\right) - a_i| = |a_i - a_i| = 0.
	\end{equation}
	\underline{Case 2}: $x \in (c_i - \delta, c_i + \delta)$. Then
	\begin{align}
		|\psi(x) - g(x)| &= \left|\frac{a_i + a_{i-1}}{2} + \left(\frac{a_i - a_{i-1}}{2\delta}\right)(x-c_i) - \psi(x)\right| \\
		&< \left|\frac{a_i + a_{i-1}}{2} + \left(\frac{a_i - a_{i-1}}{2\delta}\right)\delta - \psi(x)\right| \\
		&= \left|\frac{a_i + a_{i-1}}{2} + \frac{a_i + a_{i-1}}{2} - \psi(x)\right| \\
		&= |a_i - \psi(x)| \\
		&= 0 \textrm{ or } |a_{i+1} - a_i|.
	\end{align}
	\underline{Case 3}: $x \in (c_n - \delta, b)$. Then
	\begin{align}
		|\psi(x) - g(x)| &= \left|-\frac{a_n}{2\delta}(x-b) - \psi(x)\right| \\
		&< \left|-\frac{a_n}{2\delta}\delta - \psi(x)\right| \\
		&= \left|-\frac{a_n}{2} - \psi(x)\right| \\
		&= \frac{3a_n}{2}.
	\end{align}
	So in all three cases, we have that either $|g(x) - \psi(x)| = 0$, or $|g(x) - \psi(x)| < \frac{3a}{2}$, or $|g(x) - \psi(x)|< |a_{i+1} - a_i|$.
	
	Define the set $E$ to be the collection of points in $[a, b]$ for which $|\psi(x) - g(x)| \neq 0$. Then
	\begin{equation}
		E := \bigcup_{k=1}^{n} (c_k - \delta, c_k + \delta).
	\end{equation}
	By definition, $\forall x \in E^c$, 
	\begin{equation}
		|\psi(x) - g(x)| = 0 < \epsilon.
	\end{equation}
	Since $E$ is a countable union of intervals, we have
	\begin{align}
		m(E) &= m\left[\bigcup_{k=1}^n (c_k - \delta, c_k + \delta)\right] \\
		&\leq \sum_{k=1}^n m\left[(c_k - \delta, c_k + \delta)\right] \\
		&= \sum_{k=1}^n \ell (c_k - \delta, c_k + \delta) \\
		&= \sum_{k=1}^n 2\delta\\
		&= 2n\delta \\
		& < 2n \frac{\epsilon}{2n} \\
		&= \epsilon,
	\end{align}
	as desired.
\end{proof}
	
\end{document}
