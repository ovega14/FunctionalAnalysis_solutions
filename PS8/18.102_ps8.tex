\documentclass{article}
\usepackage[utf8]{inputenc}
\usepackage[english]{babel}
\usepackage[]{amsthm} %lets us use \begin{proof}
\usepackage[]{amssymb} %gives us the character \varnothing
\usepackage[]{amsmath}
\usepackage[parfill]{parskip} %avoid indent when skipping lines
\usepackage[toc,page]{appendix}
\usepackage{mathtools}
\usepackage{bbm}
\usepackage{hyperref}
\hypersetup{
	colorlinks=true,
	linkcolor=blue,
	filecolor=magenta,      
	urlcolor=cyan,
	pdftitle={18.102-ps8},
	pdfpagemode=FullScreen,
}
%\urlstyle{same}
\newcommand{\R}{\mathbb{R}} %the real numbers
\newcommand{\N}{\mathbb{N}} %the natural numbers
\newcommand{\M}{\mathcal{M}} %set of all Lebesgue-measurable sets
\newcommand{\Q}{\mathbb{Q}} % the rational numbers
\newcommand{\C}{\mathbb{C}} %the complex numbers

\title{18.102 Assignment 8}
\author{Octavio Vega}
\date\today

\begin{document}
\maketitle
	
\section*{Problem 1}
\subsection*{(a)}
\begin{proof}
	($\Rightarrow$) Suppose $w \in \overline{W}$. Then since $w \in H$ and since $\{e_n\}_n \subset H$ is a countably infinite orthonormal subset, we have
	\begin{equation}
		w = \sum_{n=1}^{\infty} \langle u, e_n \rangle e_n.
	\end{equation}
	Computing the norm gives
	\begin{align}
		||w||^2 &= \left\langle \sum_{n=1}^{\infty} \langle u, e_n \rangle e_n, \sum_{k=1}^{\infty} \langle u, e_k \rangle e_k \right \rangle \\
		&= \sum_{n=1}^{\infty}\sum_{k=1}^{\infty} \langle u, e_n \rangle \overline{\langle u, e_k \rangle} \langle e_n, e_k \rangle \\
		&= \sum_{n=1}^{\infty}\sum_{k=1}^{\infty} \langle u, e_n \rangle \overline{\langle u, e_k \rangle} \delta_{nk} \\
		&= \sum_{n=1}^{\infty}\langle u, e_n \rangle \overline{\langle u, e_n \rangle} \\
		&= \sum_{n=1}^{\infty} |\langle u, e_n \rangle|^2.
	\end{align}
	Thus, 
	\begin{equation}
		||w|| = \left(\sum_{n=1}^{\infty}|\langle u, e_n \rangle|^2\right)^\frac{1}{2} < \infty,
	\end{equation}
	so defining $c_n := \langle u, e_n \rangle$ yields a sequence $\{c_n\}_n \in \ell^2 (\N)$, as desired.
	
	($\Leftarrow$) Let $\{c_n\}_{n=1}^{\infty} \in \ell^{2}$ such that $w = \sum_{n=1}^{\infty} c_n e_n$. 
	
	Define $w_N := \sum_{n=1}^N c_n e_n$. Then $w = \lim\limits_{N \to \infty} w_N$, and for each $N \in N$, $w_N \in W$ since it is a finite linear combination of elements in $\{e_n\}_n$.
	
	Thus, since $\overline{W}$ contains all the limit points of $W$, then $w \in \overline{W}$, as desired.
\end{proof}

\subsection*{(b)}
\begin{proof}
	Let $w \in \overline{W}$ and $u \in H$. Then by \textbf{(a)}, we may write $w = \sum_{n=1}^{\infty} c_n e_n$ for $\{c_n\}_n \in \ell^2(\N)$. Suppose $c_n = \langle u, e_n \rangle$. Then
	\begin{align}
		||u - \sum_{n=1}^{\infty} \langle u, e_n \rangle e_n|| &= ||u - \sum_{n=1}^{\infty} c_n e_n || \\
		&= ||u - v||.
	\end{align}
	Now suppose $c_n \neq \langle u, e_n \rangle$. Then we compute
	\begin{align}
		||u - w||^2 &= \left\|u - \sum_{n=1}^{\infty} c_n e_n \right\|^2 \\
		&= \left\|u - \sum_{n=1}^{\infty} \langle u, e_n \rangle e_n + \sum_{n=1}^{\infty} \langle u, e_n \rangle e_n - \sum_{n=1}^{\infty}c_n e_n\right\|^2 \\
		&= \left\|u - \sum_{n=1}^{\infty} \langle u, e_n \rangle e_n + \sum_{n=1}^{\infty}(\langle u, e_n \rangle - c_n)e_n\right\|^2 \\
		& \geq \left\|u - \sum_{n=1}^{\infty} \langle u, e_n \rangle e_n \right\|^2.
	\end{align}
	Therefore, $\left\|u - \sum_{n=1}^{\infty} \langle u, e_n \rangle e_n\right\| \leq ||u - w||$, with equality only if \\$w = \sum_{n=1}^{\infty} \langle u, e_n \rangle e_n$, and we are done.
\end{proof}
%%%%%%%%%%%%%%%%%%%%%%%%%%%%%%%%%%%%%%%%%%%%%%%%%%%%%%%%%%%%%%%%%%%%%%%%%%%%%%%%%%%%%%%%%%%%%%%%%%%%%%%%%%%%%%%%%%%%%%%%%%%%%%%%%%%%%%%%%%%
\section*{Problem 2}
\subsection*{(a)}
\begin{proof}
	Let $\{u_k\}_k \subset W^{\perp}$ be a sequence such that $u_k \to u$ as $k \to \infty$. Then $\forall k \in \N$ and $\forall w \in W$, we have $\langle u_k, w \rangle$ = 0.
	
	By continuity of the inner product, 
	\begin{align}
		0 &= \lim\limits_{k \to \infty} \langle u_k, w\rangle \\
		&= \left\langle \lim\limits_{k \to \infty} u_k, w \right \rangle \\
		&= \langle u, w \rangle.
	\end{align}
	Thus, $u \in W^\perp$, so $W^{\perp} \subset H$ is closed.
\end{proof}

\subsection*{(b)}
\begin{proof}
	We note that
	\begin{equation}
		\left(W^\perp\right)^\perp := \{v \in H \;|\; \langle v, u \rangle = 0 \: \forall u \in W^\perp\}.
	\end{equation}
	Let $w \in W$. Then $\forall u \in W^\perp$, 
	\begin{align}
		&\langle w, u \rangle = 0 \\
		&\implies w \in \left(W^\perp\right)^\perp \\
		&\implies W \subseteq \left(W^\perp\right)^\perp .
	\end{align}
	Since $\left(W^\perp\right)^\perp$ is an orthogonal complement, then by \textbf{(a)} it is a closed linear subspace of $H$. So, $\left(W^\perp\right)^\perp$ must also be complete.
	
	Let $\{v_n\}_n \subset W$ be a Cauchy sequence. Since $W \subseteq \left(W^\perp\right)^\perp$, then $\{v_n\}_n \subset \left(W^\perp\right)^\perp$, so $\{v_n\}_n$ converges in $\left(W^\perp\right)^\perp$, i.e.
	\begin{equation}
		\lim\limits_{n \to \infty} v_n = v \in \left(W^\perp\right)^\perp.
	\end{equation}
	Thus the Cauchy sequence $\{v_n\}_n$ converges to $v$ in $\left(W^\perp\right)^\perp$.
	
	We conclude that the closure $\overline{W} = \left(W^\perp\right)^\perp$.
\end{proof}
%%%%%%%%%%%%%%%%%%%%%%%%%%%%%%%%%%%%%%%%%%%%%%%%%%%%%%%%%%%%%%%%%%%%%%%%%%%%%%%%%%%%%%%%%%%%%%%%%%%%%%%%%%%%%%%%%%%%%%%%%%%%%%%%%%%%%%%%%%%
\section*{Problem 3}
\subsection*{(a)}
\begin{proof}
	Since $f \in C^k([-\pi, \pi])$, then $f$ must be bounded. Let $B \geq 0$ be such a bound for $f$. Then we have
	\begin{align}
		\|f\|_{L^2}^2 &= \int_{-\pi}^\pi |f(t)|^2 \mathrm{d}t \\
		& \leq \int_{-\pi}^\pi B^2 \mathrm{d}t \\
		&= 2\pi B^2 \\
		& < \infty.
	\end{align}
	Thus $f \in L^2([-\pi, \pi])$.
	
	Next, we prove the following claim:
	
	\underline{Claim}: Let $k \in \N$. For each $j \in \{0, 1, ..., k\}$, 
	\begin{equation}
		\hat{f}(n) = \left(-\frac{i}{n}\right)^j \widehat{f^{(j)}}(n).
	\end{equation}
	\begin{proof}[Proof of claim:]
		(By induction on $j$).
		
		\underline{Base case}: ($j=0$) $$\hat{f}(n) = \widehat{f^{(0)}}(n).$$
		
		\underline{Inductive step:} Assume $\hat{f}(n) = \left(-\frac{i}{n}\right)^j \hat{f^{(j)}}(n)$. Then integrating by parts, we get
		\begin{align}
			\hat{f}(n) &= \left(-\frac{i}{n}\right)^j\frac{1}{2 \pi} \int_{-\pi}^\pi f^{(j)}(t) e^{-int} \mathrm{d}t \\
			&= \left(-\frac{i}{n}\right)^j\frac{1}{2 \pi}\left[f^{(j)}(t) \frac{e^{-int}}{(-in)} \Big|_{-\pi}^\pi - \int_{-\pi}^\pi f^{(j+1)}(t)\frac{e^{-int}}{(-in)} \mathrm{d}t\right] \\
			&= 0 - \frac{i}{n}\left(-\frac{i}{n}\right)^j \frac{1}{2\pi} \int_{-\pi}^\pi f^{(j+1)}e^{-int} \mathrm{d}t \\
			&= \left(-\frac{i}{n}\right)^{j+1} \widehat{f^{(j+1)}}(n),
		\end{align} 
		and the claim is proven.
	\end{proof}
	So for each $j \in \{0, 1, ..., k\}$, 
	\begin{align}
		\left|\hat{f}(n)\right| &= \left|\left(-i\right)^j \frac{1}{n^j} \widehat{f^{(j)}}(n)\right| \\
		&= \frac{1}{n^j} \left|\widehat{f^{(j)}}(n)\right|.
	\end{align}
	Let $0 \leq s \leq k$ and $N \in \N$. Then
	\begin{align}
		\sum_{|n| \leq N} \left|\hat{f}(n)\right|^2\left(1 + |n|^2\right)^s &= \sum_{|n| \leq N} \frac{1}{n^{2j}} \left|\widehat{f^{(j)}}(n)\right|^2 \left(1 + |n|^2\right)^s \\
		&= \sum_{|n| \leq N} \frac{\left(1 + |n|^2\right)^s}{n^{2j}} \left|\widehat{f^{(j)}}(n)\right|^2.
	\end{align}
	Since this holds $\forall 0 \leq j \leq k$, we can choose $j = k$, giving
	\begin{align}
		\sum_{|n| \leq N} \left|\hat{f}(n)\right|^2\left(1 + |n|^2\right)^s &= \sum_{|n| \leq N} \frac{\left(1 + |n|^2\right)^s}{n^{2k + 2}} \left|\widehat{f^{(k)}}(n)\right|^2\\
		&\leq \sum_{|n| \leq N} \left|\widehat{f^{(k)}}(n)\right|^2 \\
		& < \infty.
	\end{align}
	Sending $N \to \infty$, we get the desired result.
	
	Therefore $f \in H^s(\mathbb{T})$ $\forall 0 \leq s \leq k$.
\end{proof}

\end{document}
