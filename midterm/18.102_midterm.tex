\documentclass{article}
\usepackage[utf8]{inputenc}
\usepackage[english]{babel}
\usepackage[]{amsthm} %lets us use \begin{proof}
\usepackage[]{amssymb} %gives us the character \varnothing
\usepackage[]{amsmath}
\usepackage[parfill]{parskip} %avoid indent when skipping lines
\usepackage[toc,page]{appendix}
\usepackage{mathtools}
\usepackage{dsfont}
\usepackage{hyperref}
\hypersetup{
	colorlinks=true,
	linkcolor=blue,
	filecolor=magenta,      
	urlcolor=cyan,
	pdftitle={18.102-midterm},
	pdfpagemode=FullScreen,
}
%\urlstyle{same}
\newcommand{\R}{\mathbb{R}} %the real numbers
\newcommand{\N}{\mathbb{N}} %the natural numbers
\newcommand{\M}{\mathcal{M}} %set of all Lebesgue-measurable sets
\newcommand{\Q}{\mathbb{Q}} % the rational numbers

\title{18.102 Midterm}
\author{Octavio Vega}
\date\today

\begin{document}
\maketitle
	
\section*{Problem 1}
\begin{proof}
	We will show that $\Lambda([a, b])$ is a proper closed subspace of $C([a, b])$, which we know is a Banach space.
	Let $\{f_n\}_n$ be a cauchy sequence in $\Lambda([a, b])$ such that $f_n \to f$ pointwise. Then for every $\epsilon > 0$, $\exists N \in \N$ such that $\forall n \geq N$, $||f - f_n|| < \epsilon$. This is equivalent to
	\begin{equation}
		\sup_{x \in [a, b]} |f(x) - f_n(x)| + \sup_{x \neq y \in [a, b]}\frac{|f(x) - f_n(x) - f(y) + f_n(y)|}{|x-y|} < \epsilon.
	\end{equation}
	Since both terms on the left hand side are non-negative, this implies
	\begin{equation}
		\sup_{x \neq y \in [a, b]}\frac{|f(x) - f_n(x) - f(y) + f_n(y)|}{|x-y|} < \epsilon.
	\end{equation}
	Then for any $x \neq y \in [a, b]$, we have
	\begin{equation}
		|f(x) - f_n(x) - f(y) + f_n(y)| < \epsilon|x-y|,
	\end{equation}
	which confirms that for each $n \geq N$, the function $f - f_n$ is Lipschitz continuous. By assumtion, $f_n$ is Lipschitz continuous $\forall n \in \N$, and the sum of Lipschitz continuous functions is also Lipschitz, thus $f = f_n + (f - f_n)$ is Lipschitz continuous.
	
	So, $\lim_{n \to \infty}f_n = f \in \Lambda([a, b])$, which proves that $\Lambda([a, b])$ is a proper closed subspace of $C([a, b])$.
	
	Therefore, $\Lambda([a, b])$ is a Banach space.  
\end{proof}
%%%%%%%%%%%%%%%%%%%%%%%%%%%%%%%%%%%%%%%%%%%%%%%%%%%%%%%%%%%%%%%%%%%%%%%%%%%%%%%%%%%%%%%%%%%%%%%%%%%%%%%%%%%%%%%%%%%%%%%%%%%%%%%%%%%%%%%%%%%%
\section*{Problem 2}
\begin{proof}
	First we show that $||a + c_0||_{\ell^{\infty} / c_0} \leq \limsup_{n \to \infty} |a_n|$.
	
	Let $a = \{a_n\}_n \in \ell^{\infty}$. For each $n \in \N$, let $b_n = (a_1, a_2, ..., a_n, 0, 0, ...) \in c_0$. Then 
	\begin{align}
		\inf_{b \in c_0} ||a + b||_{\infty} & \leq \inf_n ||a - b_n||_{\infty} \\
		&= \inf_n \sup_{m \in \N} |a_m - b_m| \\
		&= \inf_n \sup_{m \geq n} |a_m| \\
		&= \limsup_{n \to \infty} |a_n|.
	\end{align}
	Thus, 
	\begin{equation}
		||a + c_0||_{\ell^{\infty} / c_0} \leq \limsup_{n \to \infty} |a_n|.
	\end{equation}
	Let $b = (b_1, b_2, b_3, ...) \in c_0$. Then for every $\epsilon > 0$, $\exists n \in N$ such that $\forall m \geq n$, $|b_m| < \epsilon$, so 
	\begin{align}
		||a + b||_{\infty} & \geq \sup_{m \geq n}|a_m| - \epsilon \\
		&\geq \limsup_{n \to \infty} |a_n| - \epsilon,
	\end{align}
	hence $\limsup_{n \to \infty}|a_n| < ||a + c_0||_{\ell^{\infty}/c_0} + \epsilon$.
	
	Therefore, $||a + c_0||_{\ell^{\infty} / c_0} = \limsup_{n \to \infty} |a_n|$.
\end{proof}
%%%%%%%%%%%%%%%%%%%%%%%%%%%%%%%%%%%%%%%%%%%%%%%%%%%%%%%%%%%%%%%%%%%%%%%%%%%%%%%%%%%%%%%%%%%%%%%%%%%%%%%%%%%%%%%%%%%%%%%%%%%%%%%%%%%%%%%%%%%%
\section*{Problem 3}
\subsection*{(a)}
\begin{proof}
	Since $\lim_{n \to \infty}T_n x = Tx$, then for every $\epsilon > 0$ $\exists N \in N$ such that $\forall n \geq N$,
	\begin{equation}
		||T_nx - Tx|| < \epsilon.
	\end{equation}
	By linearity of $T$, this is equivalent to 
	\begin{equation}
		||(T_n - T)x|| < \epsilon.
	\end{equation}
	Choose $\epsilon = ||x||$. With a sufficiently large choice of $N$, we have $\forall n \geq N$ and $\forall x \in V$,
	\begin{equation}
		||(T_n - T)x|| < ||x||.
	\end{equation}
	The above equation implies that the operator $T_n - T$ is continuous. Since $\{T_n\}_n$ is assumed to be a sequence in $\mathcal{B}(V, W)$, then $T_n - (T_n - T) = T$ is continuous.
	
	Therefore, $T$ is a bounded linear operator.
\end{proof}

\subsection*{(b)}
\begin{proof}
	Since $V$ is a Banach space with respect to both norms $||\cdot||_1$ and $||\cdot||_2$, we may regard the spaces $V_1 := (V, ||\cdot||_1)$ and $V_2 := (V, ||\cdot||_2)$ as separate Banach spaces.
	
	Consider the identity mapping $\mathds{1} \in \mathcal{B}(V_1, V_2)$. Since $\mathds{1}$ is a bounded linear operator, then $\exists C > 0$ such that $\forall v \in V_1$,
	\begin{equation}
		||v||_2 = ||\mathds{1} v||_2 \leq C ||v||_1,
	\end{equation}
	and we are done.
\end{proof}
%%%%%%%%%%%%%%%%%%%%%%%%%%%%%%%%%%%%%%%%%%%%%%%%%%%%%%%%%%%%%%%%%%%%%%%%%%%%%%%%%%%%%%%%%%%%%%%%%%%%%%%%%%%%%%%%%%%%%%%%%%%%%%%%%%%%%%%%%%%%
\section*{Problem 4}
\subsection*{(a)}
\begin{proof}
	For each $n \in \N$, define the set $F_n \subset E$ via 
	\begin{equation}
		F_n := \left\{x \in E \;|\; |f(x)| > ||f||_{\infty} + \frac{1}{n}\right\}.
	\end{equation}
	Then by definition of the essential supremum of $f$, $\forall n \in \N$, $m(F_n) = 0$. So for almost every $x \in E$ (i.e. $\forall x \in E \backslash F_n$), we have
	\begin{equation}
		|f(x)| \leq ||f||_{\infty} + \frac{1}{n}.
	\end{equation}
	Now consider $\bigcup_{n \in \N}F_n$. Since $\forall n \in \N$ we have $F_{n+1} \subset F_n$, then $(E \backslash F_n) \subset (E \backslash F_{n+1})$.
	
	By continuity from below (proved in \href{https://github.com/ovega14/FunctionalAnalysis_solutions/blob/main/PS5/18.102_ps5.pdf}{PS5.1b}), we have that
	\begin{align}
		m\left(E \backslash \bigcup_n F_n\right) &= m\left(\bigcap_n F_n^c\right) \\
		&= \lim_{n \to \infty} m(F_n^c) \\
		&= \lim_{n \to \infty} m(E \backslash F_n) \\
		&= m(E) - \lim_{n \to \infty}m(F_n) \\
		&= m(E).
	\end{align}
	This is equivalent to the statement
	\begin{equation}
		m\left(\bigcup_n F_n\right) = 0.
	\end{equation}
	Therefore, $|f(x)| \leq ||f||_{\infty}$ almost everywhere on $E$.
\end{proof}

\end{document}
