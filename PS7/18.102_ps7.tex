\documentclass{article}
\usepackage[utf8]{inputenc}
\usepackage[english]{babel}
\usepackage[]{amsthm} %lets us use \begin{proof}
\usepackage[]{amssymb} %gives us the character \varnothing
\usepackage[]{amsmath}
\usepackage[parfill]{parskip} %avoid indent when skipping lines
\usepackage[toc,page]{appendix}
\usepackage{mathtools}
\usepackage{bbm}
\usepackage{hyperref}
\hypersetup{
	colorlinks=true,
	linkcolor=blue,
	filecolor=magenta,      
	urlcolor=cyan,
	pdftitle={18.102-ps7},
	pdfpagemode=FullScreen,
}
%\urlstyle{same}
\newcommand{\R}{\mathbb{R}} %the real numbers
\newcommand{\N}{\mathbb{N}} %the natural numbers
\newcommand{\M}{\mathcal{M}} %set of all Lebesgue-measurable sets
\newcommand{\Q}{\mathbb{Q}} % the rational numbers
\newcommand{\C}{\mathbb{C}} %the complex numbers

\title{18.102 Assignment 7}
\author{Octavio Vega}
\date\today

\begin{document}
\maketitle
	
\section*{Problem 1}
\subsection*{(a)}
\begin{proof}
	Let $E = [a, b]$ and let $f: E \to \C$. Suppose $f \in L^p([a, b])$. Then
	\begin{align}
		&\Rightarrow \int_E |f|^p < \infty \\
		&\Rightarrow \left(\int_E |f|^p\right)^\frac{1}{p} < \infty \\
		&\iff ||f||_{L^p(E)} < \infty.
	\end{align}
	Now let $1 \leq q \leq p$. Then by Hölder's inequality, since $1: E \to \C$ is measurable, then
	\begin{align}
		||f||_{L^q(E)}^q &= \int_E |f|^q \\
		&= || |f|^q||_{L^1(E)} \\
		&= || 1 \cdot |f|^q||_{L^1(E)} \\
		& \leq ||1||_{L^{\frac{p}{p-q}}(E)} || |f|^q||_{L^{\frac{p}{q}}(E)} \\
		&= \left(\int_E 1\right)^{\frac{p-q}{p}} \left(\int_E \Big||f|^q\Big|^{\frac{p}{q}}\right)^{\frac{q}{p}} \\
		&= (b - a)^{\frac{p-q}{p}} \left(\int_E |f|^p\right)^{\frac{q}{p}} \\
		&= (b - a)^{\frac{p-q}{p}} ||f||_{L^p(E)}^q \\
		&< \infty.
	\end{align}
	Hence, $f \in L^p([a, b]) \Rightarrow f \in L^q ([a, b])$.
	
	Therefore $L^p([a, b]) \subset L^q([a, b])$. 
\end{proof}

\subsection*{(b)}
\begin{proof}
	Let $f \in L^p([a, b])$ and $\epsilon > 0$. Choose $N$ such that
	\begin{equation}
		||f - f \chi_{[-f^{-1}(N), f^{-1}(N)]}||_p < \frac{\epsilon}{2}.
	\end{equation}
	Let $f_n =f \chi_{[-f^{-1}(n), f^{-1}(n)]}$. From \href{https://github.com/ovega14/FunctionalAnalysis_solutions/blob/main/PS6/18.102_ps6.pdf}{PS6.2}, Littlewood's third principle tells us that "every measurable function is nearly continuous." This gives us a closed set $F$ such that
	\begin{equation}
		m([a, b] \backslash F) < \left(\frac{\epsilon}{4N}\right)^p,
	\end{equation}
	and the restriction $f_n \big|_F$ is continuous with $f_N(a) = f_N(b) = 0$. So, we have
	\begin{align}
		||f_n - f||^p &= \int_{[a, b] \backslash F} |f_n - f|_p ^p \\
		& \leq (2N)^p m([a, b]\backslash F) \\
		&< (2N)^p \frac{\epsilon^p}{(4N)^p} \\
		&= \left(\frac{\epsilon}{2} \right)^p,
	\end{align}
	i.e. $||f_n - f||_p < \epsilon$.
	
	Since $\chi$ is a step function, the theorem given in the assignment tells us that we can find a $g \in C([a, b])$ with $g(a) = g(b) = 0$. Let $g = \lim_{n \to \infty} f_n$. Then $|f - g| < \epsilon$. Thus, $C([a, b])$ is dense in $L^p([a, b])$.
	
	Therefore $L^p([a, b])$ is separable.
\end{proof}
	
\subsection*{(c)}
\begin{proof}
	For each $n \in \N$, define $f_n := f \chi_{[-n, n]}$. Since $f \in L^p(\R)$, then $f\in L^p([-n, n])$, so by the given theorem, $f_n$ is a step function and $\exists g_n \in C([-n, n])$ with $g(-n) = g(n) = 0$ such that
	\begin{equation}
		||f - f_n||_p + ||f - g_n||_p < \epsilon.
	\end{equation}
	We compute
	\begin{align}
		||f - f_n||_p ^p &= \int_{\R} |f - f_n|^p \\
		&= \int_{\R} |f|^p |1 - \chi_{[-n, n]}|^p \\
		&= \int_{\R \backslash [-n, n]} |f|^p |1 - \chi_{[-n, n]}|^p + \int_{[-n, n]} |f|^p |\chi_{[-n, n]}|^p \\
		&= \int_{\R \backslash [-n, n]} |f|^p.
	\end{align}
	Sending $n \to \infty$, we get
	\begin{equation}
		||f - f_n||_p ^p = \int_{\varnothing}|f|^p = 0.
	\end{equation}
	Thus, $||f_n - f||_p \to 0$ as $n \to \infty$. Choosing $R$ sufficiently large, we are left with $||f - g_n||_p < \epsilon$ as $n \to \infty$. Take $g = \lim_{n \to \infty} g_n$, and we are done.
	
	Therefore $||f - g||_p < \epsilon$.
\end{proof}

\subsection*{(d)}
\begin{proof}
	From part \textbf{(c)}, we know that for every $f \in L^p(\R)$ and $\epsilon > 0$, we can find a $g \in C(\R)$ such that $\forall |x| > R$, $g(x)=0$ and $||f - g||_p < \epsilon$. So, $C(\R)$ is dense in $L^p(\R)$.
	
	Let $\{g_n\}_n \subset C(\R)$. Then we are done, since this sequence is countable.
	
	Therefore $L^p(\R)$ is separable.
\end{proof}
%%%%%%%%%%%%%%%%%%%%%%%%%%%%%%%%%%%%%%%%%%%%%%%%%%%%%%%%%%%%%%%%%%%%%%%%%%%%%%%%%%%%%%%%%%%%%%%%%%%%%%%%%%%%%%%%%%%%%%%%%%%%%%%%%%%%%%%%%%%%
\section*{Problem 2}
\subsection*{(a)}
\begin{proof}
	Let $f \in L^{\infty}(E)$. The norm is given by the essential supremum, i.e.
	\begin{equation}
		||f||_{\infty} = \inf \{C \geq 0 \;|\; |f(x)| \leq C \textrm{ a.e.}\}
	\end{equation}
	In problem 4b of the \href{https://github.com/ovega14/FunctionalAnalysis_solutions/blob/main/midterm/18.102_midterm.pdf}{midterm}, we showed that the ess. sup satisfied homogeneity and the triangle inequality. In 4a, we showed that $||f||_{\infty} \geq |f(x)|$ almost everywhere on $E$, so if $||f||_{\infty} = 0$, then $f=0$ a.e., so $||\cdot||_{\infty}$ is a well-defined norm. Thus, $L^{\infty}(E)$ is a normed vector space.
	
	Let $\epsilon > 0$. Then $\exists N_0 \in \N$ such that $\forall n \geq N_0$ and $x \in F$, $|f_n(x) - f(x)| < \epsilon$. Then 
	\begin{equation}
		m(\{x \in E \;|\; |f_n(x) - f(x)| > \epsilon\}) = 0.
	\end{equation} 
	So $\forall n \geq N_0$, $||f_n - f|| < \epsilon$. Thus if $\exists F \subset E$ such that $m(F^c) = 0$ and $f_n \to f$ uniformly on $F$, then $||f_n - f|| \to 0$.
	
	Let $\{f_n\}_n \subset L^{\infty}(E)$ be a Cauchy sequence. Then for $\epsilon > 0$, $\exists N_0(\epsilon) \in \N$ such that $\forall n, m \geq N_0(\epsilon)$, we have $||f_n - f_m|| < \frac{\epsilon}{2}$. Then $\exists F^{\epsilon} \subset E$ such that $m(F^{\epsilon}) = 0$ and for $x \notin F^{\epsilon}$, $|f_n(x) - f_m(x)| < \frac{\epsilon}{2}$.
	
	Let $F = \cup_n F^{\frac{1}{n}}$. Then $m(F) = 0$ and $\forall x \in F^c$, $\{f_n(x)\}_n$ is Cauchy. Let $f(x) = \lim_{n \to \infty} f_n(x)$ for $x \in F^c$, and let $\epsilon > 0$. Then $\exists k_0 \in \N$ such that $\forall k \geq k_0$, $\frac{1}{k} < \frac{\epsilon}{2}$. Then $\forall n, m \geq N_0(k_0)$, we have
	\begin{equation}
		|f_n(x) - f_m(x)| < \frac{1}{k} < \frac{\epsilon}{2}.
	\end{equation}
	Sending $m \to \infty$ gives us $|f_n(x) - f(x)| < \epsilon$, hence $||f_n - f|| \to 0$.
	
	Therefore $L^{\infty}(E)$ is a Banach space.
\end{proof}

\subsection*{(b)}
\begin{proof}
	Let $f \in L^{\infty}([a, b])$. We have
	\begin{align}
		||f||_{L^{\infty}([a, b])} &= \inf \{C \geq 0 \;|\; m({x \in [a, b] \;|\; |f(x)| > C}) = 0\} \\
		&= \inf \{C \geq 0 \;|\; |f(x)| \leq C \textrm{ a.e. on } [a, b]\}.
	\end{align}
	Note that $\forall x \in [a, b]$, $|f(x)| \leq \sup_{x \in [a, b]} |f(x)|$, so
	\begin{align}
		||f||_{L^{\infty}([a, b])} &= \inf \{C \geq 0 \;|\; C \geq \sup_{x \in [a, b]} |f(x)|\} \\
		&= \sup_{x \in [a, b]} |f(x)| \\
		&= ||f||_{\infty}.
	\end{align}
	Consider the function $f = \chi_{[a, b]}$. Then 
	\begin{equation}
		||f||_{\infty} = \sup_{x \in [a, b]} |\chi_{[a, b]}(x)| = 1,
	\end{equation}
	so $f \in L^{\infty}([a, b])$. Now suppose $\exists \{f_n\}_n \subset C([a, b])$ such that $||f_n - f||_{\infty} \to 0$. Then for every $\epsilon > 0$, $\exists N \in \N$ such that $\forall n \geq N$, $||f_n - f||_{\infty} < \epsilon$. Choose $\epsilon = \frac{1}{2}$. Then $\exists N$ such that $\forall n \geq N$, $||f_n - f||_{\infty} < \frac{1}{2}$. In particular, $\exists N$ such that $||f_n - f||_{\infty} < \frac{1}{2}$ and such that $|f_n(x) - f(x)| > \frac{1}{2}$ for some $x \in [a, b]$. But this contradicts our conclusion above that $||f||_{\infty} = ||f||_{L^{\infty}([a, b])}$, so this $f$ cannot be approximated arbitrarily closely by continuous functions.
	
	Therefore $C([a, b])$ is not dense in $L^{\infty}([a, b])$.
\end{proof}
%%%%%%%%%%%%%%%%%%%%%%%%%%%%%%%%%%%%%%%%%%%%%%%%%%%%%%%%%%%%%%%%%%%%%%%%%%%%%%%%%%%%%%%%%%%%%%%%%%%%%%%%%%%%%%%%%%%%%%%%%%%%%%%%%%%%%%%%%%%%
\section*{Problem 3}
\begin{proof}
	First, we show that $T: L^{p}([a, b]) \to L^{p}([a, b])$, defined via
	\begin{equation}
		Tf(x) := f(x)g(x)
	\end{equation}
	for $g \in L^{\infty}([a, b])$, is linear.
	
	Let $f_1, f_2 \in L^{p}([a, b])$ and $\lambda_1, \lambda_2 \in \C$. Then
	\begin{align}
		T(\lambda_1 f_1 + \lambda_2 f_2) &= (\lambda_1 f_1 + \lambda_2 f_2)g\\
		&= \lambda_1 f_1 g + \lambda_2 f_2 g \\
		&= \lambda_1 Tf_1 + \lambda_2 Tf_2.
	\end{align}
	Thus, $T$ is linear. We can also check that $T$ does indeed map $L^p([a, b])$ into itself; let $f \in L^{p}([a, b])$. Then
	\begin{align}
		\int_a ^b |Tf|^p &= \int_a ^b |fg|^p \\
		&= \int_a ^b |f|^p |g|^p \\
		&\leq \sup_{x \in [a, b]}|g(x)|^p \int_a ^b |f|^p \\
		&= ||g||_{\infty}^p ||f||_{\infty}^p \\
		&< \infty,
	\end{align}
	hence $Tf \in L^p ([a, b])$. 
	
	Next we show that $T$ is bounded:
	\begin{align}
		||T|| &= \sup_{||f||_p = 1}||fg||_p \\
		&= \sup_{||f||_p=1}\left(\int_a ^b |fg|^p\right)^\frac{1}{p} \\
		&= \sup_{||f||_p = 1} \left(||g||_{\infty}^p \int_a ^b |f|^p\right)^\frac{1}{p} \\
		&= \sup_{||f||_p = 1} ||g||_{\infty} ||f||_p \\
		&= ||g||_{L^{\infty}}.
	\end{align}
	Thus we conclude that $T \in \mathcal{B}(L^p([a, b]), L^p([a, b]))$ with $||T|| = ||g||_{L^{\infty}}$.
\end{proof}
%%%%%%%%%%%%%%%%%%%%%%%%%%%%%%%%%%%%%%%%%%%%%%%%%%%%%%%%%%%%%%%%%%%%%%%%%%%%%%%%%%%%%%%%%%%%%%%%%%%%%%%%%%%%%%%%%%%%%%%%%%%%%%%%%%%%%%%%%%%%
\section*{Problem 4}
\subsection*{(a)}
\begin{proof}
	We compute and expand:
	\begin{align}
		\frac{1}{4}[||u + v||^2 - ||u - v||^2 + i||u + iv||^2 - i ||u - iv||^2] \\= \frac{1}{4}[\langle u+v, u+v\rangle - \langle u-v, u-v\rangle + i\langle u+iv, u+iv\rangle - i\langle u-iv, u-iv\rangle] \\
		= \frac{1}{4}[2\langle u,v\rangle + 2\langle v,u\rangle + 2\langle u,v\rangle - 2\langle v,u\rangle] \\
		= \Re \langle u, v\rangle + i \Im \langle u, v\rangle \\
		= \langle u, v \rangle.
	\end{align}
	Thus, the polarization identity holds.
\end{proof}

\subsection*{(b)}
\begin{proof}
	Suppose $||\cdot||$ satisfies the parallelogram law. Then $\forall u, v \in H$, we have
	\begin{equation}
		||u + v||^2 + ||u - v||^2 = 2 ||u||^2 + 2||v||^2.
	\end{equation}
	By the polarization identity proved in part \textbf{(a)}, we defined
	\begin{equation}
		\langle u, v \rangle = \frac{1}{4}[||u + v||^2 - ||u - v||^2 + i||u + iv||^2 - i ||u - iv||^2].
	\end{equation}
	We check the following properties:
	
	(i) If $v = u$, then
	\begin{align}
		\langle u, u \rangle &= \frac{1}{4} ||2u||^2 + \frac{i}{4}||(1 + i)u||^2 - \frac{i}{4}||(1 - i)u||^2 \\
		&= ||u||^2 + \frac{\sqrt{2}}{4}i ||u||^2 - \frac{\sqrt{2}}{i}||u||^2 \\
		&= ||u||^2.
	\end{align}
	(ii) \begin{align}
		\langle v, u \rangle  &= \frac{1}{4}[||v + u||^2 - ||v - u||^2 + i ||v + iu||^2 - i||v - iu||^2] \\
		&= \frac{1}{4}[||u + v||^2 - ||u - v||^2 + i||u - iv||^2 - i||u + iv||^2] \\
		&= \overline{\langle u, v \rangle}.
	\end{align}
	(iii) Let $u, v, w \in H$. Then
	\begin{align}
		4 \Re \langle u+v, w \rangle = ||u + v + w||^2 - ||u + v - w||^2 \\
		= ||v + w ||^2 + ||u + v||^2 - ||u - w||^2 + ||u||^2 + ||w||^2 - ||v||^2 \\
		 - (||w - v||^2 + ||u - v||^2 - ||u - w||^2 + ||u||^2 + ||w||^2 - ||v||^2) \\
		= ||w + v||^2 - ||w - v||^2 + ||u + v||^2 - ||u - v||^2 \\
		= 4 \Re \langle u, w \rangle + 4 \Re \langle v, w \rangle,
	\end{align}
	and similarly for $\Im \langle u+v, w \rangle$. So $\langle\cdot, \cdot \rangle$ is linear.
	
	Therefore $\langle \cdot, \cdot \rangle$ is a Hermitian inner product on $H$.
\end{proof}

\end{document}
