\documentclass{article}
\usepackage[utf8]{inputenc}
\usepackage[english]{babel}
\usepackage[]{amsthm} %lets us use \begin{proof}
\usepackage[]{amssymb} %gives us the character \varnothing
\usepackage[]{amsmath}
\usepackage[parfill]{parskip} %avoid indent when skipping lines

\title{18.102 Assignment 2}
\author{Octavio Vega}
\date\today

\begin{document}
\maketitle
	
\section*{Problem 1}
\subsection*{(a)}
\begin{proof}
	Let $B$ be a Banach space. Suppose $T \in \mathcal{B}(B,B)$ and $||I-T||<1$. Then by Geometric series, 
	\begin{equation}
		\sum_{n=0}^{\infty} ||(I-T)^n|| \leq \sum_{n=0}^{\infty} ||I-T||^n = \frac{1}{1-||I-T||}<\infty.
	\end{equation}
	So the series $\sum_{n=0}^{\infty} (I-T)^n$ converges absolutely, which implies that it converges. Fix $m\in\mathbb{N}$. Then
	\begin{align}
		T\sum_{n=0}^m (I-T)^n &= [I-(I-T)]\sum_{n=0}^m (I-T)^n \\
		&= \sum_{n=0}^m (I-T)^n - \sum_{n=0}^m (I-T)^{n+1} \\
		&= I - (I-T)^{m+1}, \textrm{ by telescoping sum.}
	\end{align}
	By continuity of $T$, 
	\begin{align}
		T\sum_{n=0}^{\infty}(I-T)^n &= T\left(\lim_{m\to\infty}\sum_{n=0}^m (I-T)^n\right)\\
		&= \lim_{m\to\infty} T \sum_{n=0}^m (I-T)^n \\
		&= \lim_{m\to\infty}\left[I - (I-T)^{m+1}\right] \\
		&= I,
	\end{align}
	since $||I-T||<1$.
	We can similarly show that $\sum_{n=0}^{\infty}(I-T)^n = I$.
	
	Thus, $T$ is indeed invertible, and $\sum_{n=0}^{\infty} (I-T)^n \rightarrow T^{-1}$ in $\mathcal{B}(B,B)$.
\end{proof}

\subsection*{(b)}
\begin{proof}
	Let $\mathcal{I} := \{T\in\mathcal{B}(B,B) | T^{-1} \textrm{ exists}\}$. We want to show that $\forall T\in\mathcal{I}$, $\exists \delta>0$ such that if $||S-T||<\delta \implies S\in\mathcal{I}$.
	
	Choose $\delta = \frac{1}{||T^{-1}||}$, and write 
	\begin{equation}
		S = T - (T-S) = T\left[I-T^{-1}\left(T-S\right)\right].
	\end{equation}
	If $||S-T||<\delta = \frac{1}{||T^{-1}||}$, then
	\begin{align}
		\frac{1}{||T^{-1}||} &> ||S-T|| \\
		&= ||T-T\left[I-T^{-1}(T-S)\right]|| \\
		&= ||T||\cdot||I-\left[I-T^{-1}(T-S)\right]||
	\end{align}
	\begin{align}
		&\implies ||I-\left[I-T^{-1}(T-S)\right]|| < \frac{1}{||T^{-1}||\cdot||T||} = 1 \\
		&\implies ||T^{-1}(T-S)|| = ||I-T^{-1}S|| < 1.
	\end{align}
	So by (a), $T^{-1}S$ is invertible, which implies that $S$ is invertible. Thus, $\exists \delta>0$ such that if $S\in B_{\delta}(T)$, then $S\in\mathcal{I}$.
	
	Therefore, $\mathcal{I}$ is open.
\end{proof}

\section*{Problem 2}
\subsection*{(a)}
\begin{proof}
	To show that $||v + W||$ is a norm, we will show that it obeys positive definiteness, homogeneity, and the triangle inequality.
	
	First, suppose that $0 = ||v+W|| = \inf\limits_{w\in W}||v + w||$. Then since $||\cdot||_{V}$ is a norm on $V$, 
	\begin{equation}
		||w+w|| = 0 \iff v + w = 0 \implies v = -w.
	\end{equation}
	So $\exists$ a sequence $\{w_k\}_k \subset W$ such that $w_k \rightarrow -v$. Since $W$ is closed, $-v\in W \implies v\in V$. But then $v+W = 0 + W$ because $v \in W$. 
	
	Thus, $||v + W|| = 0 \iff v = 0$ (definiteness).
	
	Also, $||v+W|| = \inf\limits_{w\in W}||v+w|| \geq 0$ because $||\cdot||_V$ is a norm, and $||v+w|| \geq 0$ $\forall w\in W$.
	
	Let $\lambda \in \mathbb{K}$. Then since $\lambda W = W$, 
	\begin{align}
		||\lambda (v + W)|| &= ||\lambda v + W|| \\
		&= \inf\limits_{w\in W} ||\lambda v + w|| \\
		&= \inf\limits_{w\in W} |\lambda| \cdot ||v + \frac{w}{\lambda}|| \\
		&= |\lambda|\inf\limits_{w\in W}||v+w|| \\
		&= |\lambda|\cdot||v+W|| \quad \textrm{(homogeneity)}.
	\end{align}
	Now let $u+W$, $v+W \in V/W$. Then
	\begin{align}
		||(u+W) + (v+W)|| &= ||u+v+W|| \\
		&= \inf\limits_{w\in W}||u+v+w|| \\
		&= \inf\limits_{w\in W}||u+v+2w|| \\
		&= \inf\limits_{w\in W}||u+w + v+w|| \\
		&\leq \inf\limits_{w\in W}\left(||u+w|| + ||v+w||\right)\\
		&\leq \inf\limits_{w\in W}||u+w|| + \inf\limits_{w\in W}||v+w||\\
		&= ||u+W|| + ||v+W|| \quad \textrm{(triangle inequality)}.
	\end{align}
	Thus, $||v+W||$ is a norm on $V/W$.
\end{proof}

\subsection*{(b)}
TODO TODO TODO TODO

\section*{Problem 3}
\begin{proof}
	Let $\{v_n\}_n$ be a sequence of elements in $V$. Suppose that the series $\sum_{n}(v_n+W)$ is absolutely summable, i.e. that $\sum_{n}||v_n + W||$ converges. Since $||v_n + W|| = \inf\limits_{w\in W}||v_n + w||$, then for each $n\in \mathbb{N}$, $\exists w_n \in W$ such that 
	\begin{align}
		||v_n + w_n|| &\leq ||v_n + W|| + 2^{-n} \\
		\implies \sum_{n}||v_n + w_n|| &\leq \sum_{n}||v_n + W|| + \sum_{n}2^{-n} \\
		&= \sum_n ||v_n + W|| + 1.
	\end{align}
	Then by comparison, $\sum_n ||v_n + w_n||$ converges, so $\sum_n (v_n + w_n)$ converges.
	
	Since $V$ is a Banach space, then, by closure, $\exists v \in V$ such that \\ $v = \sum_n (v_n + w_n)$. Then
	\begin{align}
		\lim_{N\to\infty} v+ W - \sum_{n=1}^N (v_n + W) &= \sum_{n=1}^{\infty} (v_n + w_n) + W - \lim_{N\to\infty}\sum_{n=1}^N(v_n + W) \\
		&= \sum_{n=1}^{\infty}v_n + \sum_{n=1}^{\infty}w_n + W -\lim_{N\to\infty}\sum_{n=1}^N(v_n + W) \\
		&= \sum_{n=1}^{\infty}v_n + W -\lim_{N\to\infty}\sum_{n=1}^{N}(v_n + W) \\
		&= \sum_{n=1}^{\infty}v_n + W -\lim_{N\to\infty}\sum_{n=1}^{N} v_n - W \\
		&= \sum_{n=1}^{\infty}v_n - \lim_{N\to\infty}\sum_{n=1}^{N} = 0.
	\end{align}
	So $\sum_n (v_n + W) = v+W$, thus $\sum_n (v_n + W)$ converges in $V/W$.
	
	Therefore $V/W$ is a Banach space.
\end{proof}
\end{document}