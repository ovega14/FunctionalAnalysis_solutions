\documentclass{article}
\usepackage[utf8]{inputenc}
\usepackage[english]{babel}
\usepackage[]{amsthm} %lets us use \begin{proof}
\usepackage[]{amssymb} %gives us the character \varnothing
\usepackage[]{amsmath}
\usepackage[parfill]{parskip} %avoid indent when skipping lines

\title{18.102 Assignment 2}
\author{Octavio Vega}
\date\today

\begin{document}
\maketitle
	
\section*{Problem 1}
\subsection*{(a)}
\begin{proof}
	Let $B$ be a Banach space. Suppose $T \in \mathcal{B}(B,B)$ and $||I-T||<1$. Then by Geometric series, 
	\begin{equation}
		\sum_{n=0}^{\infty} ||(I-T)^n|| \leq \sum_{n=0}^{\infty} ||I-T||^n = \frac{1}{1-||I-T||}<\infty.
	\end{equation}
	So the series $\sum_{n=0}^{\infty} (I-T)^n$ converges absolutely, which implies that it converges. Fix $m\in\mathbb{N}$. Then
	\begin{align}
		T\sum_{n=0}^m (I-T)^n &= [I-(I-T)]\sum_{n=0}^m (I-T)^n \\
		&= \sum_{n=0}^m (I-T)^n - \sum_{n=0}^m (I-T)^{n+1} \\
		&= I - (I-T)^{m+1}, \textrm{ by telescoping sum.}
	\end{align}
	By continuity of $T$, 
	\begin{align}
		T\sum_{n=0}^{\infty}(I-T)^n &= T\left(\lim_{m\to\infty}\sum_{n=0}^m (I-T)^n\right)\\
		&= \lim_{m\to\infty} T \sum_{n=0}^m (I-T)^n \\
		&= \lim_{m\to\infty}\left[I - (I-T)^{m+1}\right] \\
		&= I,
	\end{align}
	since $||I-T||<1$.
	We can similarly show that $\sum_{n=0}^{\infty}(I-T)^n = I$.
	
	Thus, $T$ is indeed invertible, and $\sum_{n=0}^{\infty} (I-T)^n \rightarrow T^{-1}$ in $\mathcal{B}(B,B)$.
\end{proof}

\end{document}