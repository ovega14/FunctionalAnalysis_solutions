\documentclass{article}
\usepackage[utf8]{inputenc}
\usepackage[english]{babel}
\usepackage[]{amsthm} %lets us use \begin{proof}
\usepackage[]{amssymb} %gives us the character \varnothing
\usepackage[]{amsmath}
\usepackage[parfill]{parskip} %avoid indent when skipping lines

\title{18.102 Assignment 2}
\author{Octavio Vega}
\date\today

\begin{document}
\maketitle
	
\section*{Problem 1}
\subsection*{(a)}
\begin{proof}
	Let $B$ be a Banach space. Suppose $T \in \mathcal{B}(B,B)$ and $||I-T||<1$. Then by Geometric series, 
	\begin{equation}
		\sum_{n=0}^{\infty} ||(I-T)^n|| \leq \sum_{n=0}^{\infty} ||I-T||^n = \frac{1}{1-||I-T||}<\infty.
	\end{equation}
	So the series $\sum_{n=0}^{\infty} (I-T)^n$ converges absolutely, which implies that it converges. Fix $m\in\mathbb{N}$. Then
	\begin{align}
		T\sum_{n=0}^m (I-T)^n &= [I-(I-T)]\sum_{n=0}^m (I-T)^n \\
		&= \sum_{n=0}^m (I-T)^n - \sum_{n=0}^m (I-T)^{n+1} \\
		&= I - (I-T)^{m+1}, \textrm{ by telescoping sum.}
	\end{align}
	By continuity of $T$, 
	\begin{align}
		T\sum_{n=0}^{\infty}(I-T)^n &= T\left(\lim_{m\to\infty}\sum_{n=0}^m (I-T)^n\right)\\
		&= \lim_{m\to\infty} T \sum_{n=0}^m (I-T)^n \\
		&= \lim_{m\to\infty}\left[I - (I-T)^{m+1}\right] \\
		&= I,
	\end{align}
	since $||I-T||<1$.
	We can similarly show that $\sum_{n=0}^{\infty}(I-T)^n = I$.
	
	Thus, $T$ is indeed invertible, and $\sum_{n=0}^{\infty} (I-T)^n \rightarrow T^{-1}$ in $\mathcal{B}(B,B)$.
\end{proof}

\subsection*{(b)}
\begin{proof}
	Let $\mathcal{I} := \{T\in\mathcal{B}(B,B) | T^{-1} \textrm{ exists}\}$. We want to show that $\forall T\in\mathcal{I}$, $\exists \delta>0$ such that if $||S-T||<\delta \implies S\in\mathcal{I}$.
	
	Choose $\delta = \frac{1}{||T^{-1}||}$, and write 
	\begin{equation}
		S = T - (T-S) = T\left[I-T^{-1}\left(T-S\right)\right].
	\end{equation}
	If $||S-T||<\delta = \frac{1}{||T^{-1}||}$, then
	\begin{align}
		\frac{1}{||T^{-1}||} &> ||S-T|| \\
		&= ||T-T\left[I-T^{-1}(T-S)\right]|| \\
		&= ||T||\cdot||I-\left[I-T^{-1}(T-S)\right]||
	\end{align}
	\begin{align}
		&\implies ||I-\left[I-T^{-1}(T-S)\right]|| < \frac{1}{||T^{-1}||\cdot||T||} = 1 \\
		&\implies ||T^{-1}(T-S)|| = ||I-T^{-1}S|| < 1.
	\end{align}
	So by (a), $T^{-1}S$ is invertible, which implies that $S$ is invertible. Thus, $\exists \delta>0$ such that if $S\in B_{\delta}(T)$, then $S\in\mathcal{I}$.
	
	Therefore, $\mathcal{I}$ is open.
\end{proof}

\section*{Problem 2}
\subsection*{(a)}
\begin{proof}
	To show that $||v + W||$ is a norm, we will show that it obeys positive definiteness, homogeneity, and the triangle inequality.
	
	First, suppose that $0 = ||v+W|| = \inf\limits_{w\in W}||v + w||$. Then since $||\cdot||_{V}$ is a norm on $V$, 
	\begin{equation}
		||w+w|| = 0 \iff v + w = 0 \implies v = -w.
	\end{equation}
	So $\exists$ a sequence $\{w_k\}_k \subset W$ such that $w_k \rightarrow -v$. Since $W$ is closed, $-v\in W \implies v\in V$. But then $v+W = 0 + W$ because $v \in W$. 
	
	Thus, $||v + W|| = 0 \iff v = 0$ (definiteness).
	
	Also, $||v+W|| = \inf\limits_{w\in W}||v+w|| \geq 0$ because $||\cdot||_V$ is a norm, and $||v+w|| \geq 0$ $\forall w\in W$.
	
	Let $\lambda \in \mathbb{K}$. Then since $\lambda W = W$, 
	\begin{align}
		||\lambda (v + W)|| &= ||\lambda v + W|| \\
		&= \inf\limits_{w\in W} ||\lambda v + w|| \\
		&= \inf\limits_{w\in W} |\lambda| \cdot ||v + \frac{w}{\lambda}|| \\
		&= |\lambda|\inf\limits_{w\in W}||v+w|| \\
		&= |\lambda|\cdot||v+W|| \quad \textrm{(homogeneity)}.
	\end{align}
	Now let $u+W$, $v+W \in V/W$. Then
	\begin{align}
		||(u+W) + (v+W)|| &= ||u+v+W|| \\
		&= \inf\limits_{w\in W}||u+v+w|| \\
		&= \inf\limits_{w\in W}||u+v+2w|| \\
		&= \inf\limits_{w\in W}||u+w + v+w|| \\
		&\leq \inf\limits_{w\in W}\left(||u+w|| + ||v+w||\right)\\
		&\leq \inf\limits_{w\in W}||u+w|| + \inf\limits_{w\in W}||v+w||\\
		&= ||u+W|| + ||v+W|| \quad \textrm{(triangle inequality)}.
	\end{align}
	Thus, $||v+W||$ is a norm on $V/W$.
\end{proof}

\subsection*{(b)}
TODO TODO TODO TODO

\section*{Problem 3}
\begin{proof}
	Let $\{v_n\}_n$ be a sequence of elements in $V$. Suppose that the series $\sum_{n}(v_n+W)$ is absolutely summable, i.e. that $\sum_{n}||v_n + W||$ converges. Since $||v_n + W|| = \inf\limits_{w\in W}||v_n + w||$, then for each $n\in \mathbb{N}$, $\exists w_n \in W$ such that 
	\begin{align}
		||v_n + w_n|| &\leq ||v_n + W|| + 2^{-n} \\
		\implies \sum_{n}||v_n + w_n|| &\leq \sum_{n}||v_n + W|| + \sum_{n}2^{-n} \\
		&= \sum_n ||v_n + W|| + 1.
	\end{align}
	Then by comparison, $\sum_n ||v_n + w_n||$ converges, so $\sum_n (v_n + w_n)$ converges.
	
	Since $V$ is a Banach space, then, by closure, $\exists v \in V$ such that \\ $v = \sum_n (v_n + w_n)$. Then
	\begin{align}
		\lim_{N\to\infty} v+ W - \sum_{n=1}^N (v_n + W) &= \sum_{n=1}^{\infty} (v_n + w_n) + W - \lim_{N\to\infty}\sum_{n=1}^N(v_n + W) \\
		&= \sum_{n=1}^{\infty}v_n + \sum_{n=1}^{\infty}w_n + W -\lim_{N\to\infty}\sum_{n=1}^N(v_n + W) \\
		&= \sum_{n=1}^{\infty}v_n + W -\lim_{N\to\infty}\sum_{n=1}^{N}(v_n + W) \\
		&= \sum_{n=1}^{\infty}v_n + W -\lim_{N\to\infty}\sum_{n=1}^{N} v_n - W \\
		&= \sum_{n=1}^{\infty}v_n - \lim_{N\to\infty}\sum_{n=1}^{N} = 0.
	\end{align}
	So $\sum_n (v_n + W) = v+W$, thus $\sum_n (v_n + W)$ converges in $V/W$.
	
	Therefore $V/W$ is a Banach space.
\end{proof}

\section*{Problem 4}
\subsection*{(a)}
\begin{proof}
	Let $\{v_n\}_n$ be a sequence of elements in $\ker{(T)}$ such that $v_n \rightarrow v \in V$ and $Tv_n \rightarrow w \in W$. Then $\forall n \in \mathbb{N}$,
	\begin{align}
		&\implies Tv_n = 0 \\
		&\implies \{Tv_n\}_n \rightarrow w = 0.
	\end{align}
	By continuity of $T$, 
	\begin{equation}
		0 = \lim_{n\to\infty} Tv_n = T\left(\lim_{n\to\infty}v_n\right) = Tv,
	\end{equation}
	so $v \in \ker{(T)}$. Hence $\ker{(T)}$ is closed.
\end{proof}

\subsection*{(b)}
\begin{proof}
	($\Rightarrow$) Suppose $V/\ker{(T)}$ is isomorphic to $\textrm{range}(T)$. Then $\exists$ isomorphism $S:V/\ker{(T)} \longrightarrow \textrm{range}(T)$. We claim that the operator defined via \\ $S(v + \ker{(T)}) = Tv$ satisfies this. 
	
	First, we show that $S$ is linear. Let $v_1, v_2 \in V/\ker{(T)}$. Then by linearity of $T$,
	\begin{align}
		S(v_1 + v_2 + \ker{(T)}) &= T(v_1 + v_2) \\
		&= Tv_1 + Tv_2 \\
		&= S(v_1 + \ker{(T)}) + S(v_2 + \ker{(T)}).
	\end{align}
	Let $\lambda \in \mathbb{K}$. Then by linearity of $T$ and since $\lambda \cdot \ker{(T)} = \ker{(T)}$,
	\begin{align}
		S(\lambda (v + \ker{(T)})) &= S(\lambda v + \ker{(T)}) \\
		&= T(\lambda v) \\
		&= \lambda Tv\\
		&= \lambda S(v+ \ker{(T)}).
	\end{align}
	Thus, $S$ is linear.
	
	Next, we show that $S$ is bounded. We have
	\begin{align}
		||S|| &= \sup\limits_{||v|| = 1}||S(v + \ker{(T)})|| \\
		&= \sup\limits_{||v|| = 1}||Tv|| \\
		&= ||T||.
	\end{align}
	Thus $S$ is bounded, since $T\in\mathcal{B}(V,W)$. 
	So, $S$ is indeed an isomorphism, which confirms that $V/\ker{(T)}$ is isomorphic to $\textrm{range}(T)$.
	
	Now we proceed to the main part of the proof, where we will show that the above implies that $\textrm{range}(T)$ is closed.
	
	Note that by problems 2 and 3, the space $V/\ker{(T)}$ is a Banach space because we showed in \textbf{(a)} that $\ker{(T)}$ is a proper closed supspace of $V$, and $V$ is a Banach space. 
	
	Let $\{w_j\}_j$ be a sequence in $\textrm{range}(T)$ such that $w_j \rightarrow w \in W$. Then $\{w_j\}_{j\in\mathbb{N}}$ is Cauchy. Since $S^{-1}$ is a continuous linear operator, then $\{S^{-1}(w_j)\}_j$ is also a Cauchy sequence in $V/\ker{(T)}$. 
	
	Since $V/\ker{(T)}$ is a Banach space, then it is complete. So $\exists v\in V/\ker{(T)}$ such that
	\begin{equation}
		S^{-1}(w_j) \rightarrow v.
	\end{equation}
	By continuity, $S\left(S^{-1}(w_j)\right)\rightarrow S(v)$, then
	\begin{align}
		&\implies \lim_{j\to\infty} w_j = w = S(v)\\
		&\implies w \in \textrm{range(T)}.
	\end{align}
	Thus, $\textrm{range}(T)$ is closed in $W$.
	
	($\Leftarrow$) Suppose $\textrm{range}(T)$ is closed. Then $\textrm{range}(T)\subset W$ is a Banach space. The operator $S: V/\ker{(T)} \longrightarrow \textrm{range}(T)$ as defined before is a well-defined, bijective, bounded linear operator, i.e. $S\in\mathcal{B}(V/\ker{(T)}, \textrm{range}(T))$.
	Then by the Open Mapping theorem, $S^{-1}\in \mathcal{B}(\textrm{range}(T), V/\ker{(T)})$.
	
	Thus $S$ is an isomorphism, and we are done.
\end{proof}

\section*{Problem 5}
\subsection*{(a)}
\begin{proof}
	Let $b \in \ell^1$, $\epsilon >0$, $N \in \mathbb{N}$. Define the truncated sequence
	\begin{equation}
		a := \{b_1, b_2, b_3, ..., b_N, 0, 0, ...\}.
	\end{equation}
	Then $\sum_{k=1}^{\infty}k |a_k| = \sum_{k=1}^N k |b_k| < \infty$, so $a\in W$. We choose $N$ such that $\sum_{k=1}^N |b_k| > \sum_{k=1}^{\infty}|b_k| - \epsilon$. [Note that this is always possible since the infinite series converges, so its sequence of partial sums also converges.] Then we have
	\begin{align}
		||a-b||_1 &= \sum_{k=1}^{\infty}|a_k - b_k| \\
		&= \sum_{k=1}^N|b_k - b_k| + \sum_{k=N+1}^{\infty}|0-b_k| \\
		&= \sum_{k=N+1}^{\infty}|b_k|\\
		&= \sum_{k=1}^{\infty} |b_k| - \sum_{k=1}^N |b_k|\\
		&< \epsilon.
	\end{align}
	So we have shown that for every $\epsilon > 0$ and $b \in \ell^1$, $\exists N \in \mathbb{N}$ such that $||a-b||_1 < \epsilon$, i.e. $B(b,\epsilon)\cap W \neq \emptyset$. Thus $W$ is dense in $\ell^1$.
	
	Now consider the sequence $\{b_k\}_k$ given by $b_k = \frac{1}{k^2}$. Then $\sum_k b_k$ converges absolutely ($p>1$), but 
	\begin{equation}
		\sum_{k=1}^{\infty}k |b_k| = \sum_{k=1}^{\infty}k\cdot \frac{1}{k^2} = \sum_{k=1}^{\infty}\frac{1}{k}
	\end{equation}
	diverges by Harmonic series. So $b\in \ell^1$ but $b\notin W$. Hence $\ell^1 \neq W$.
	
	Thus we conclude that $W$ is a proper, dense subset of $\ell^1$.
\end{proof}

\subsection*{(b)}
\begin{proof}
	Define the graph of $T$ by 
	\begin{equation}
		\Gamma (T) := \{(a, Ta) \; | \; a\in W\} \subset W\times \ell^1.
	\end{equation}
	Let $\{x_n\}_n$ be a sequence in $W$. Then $Tx_n \in \ell^1$, i.e. $(x_n, Tx_n) \in \Gamma (T)$.
	
	Suppose $x_n \rightarrow x$ and $Tx_n \rightarrow y$. By definition of $T$, we have $(Tx_n)_k = k (x_n)_k$. Then
	\begin{align}
		\lim_{n\to\infty}(Tx_n)_k &= \lim_{n\to\infty} k(x_n)_k\\
		&= k \lim_{n\to\infty}(x_n)_k \\
		&= k x_k \\
		&= (Tx)_k\\
		&= y_k.
	\end{align}
	Thus $Tx=y\in \ell^1$. Also, $\lim_{n\to\infty} x_n = x = \{x_k\}_k$. Then
	\begin{align}
		&\implies \sum_k k |x_k| = \sum_k k |\lim_{n\to\infty}(x_n)_k| < \infty\\
		&\implies x \in W.
	\end{align}
	Hence, we have shown that $\begin{pmatrix} x_n\\ Tx_n \end{pmatrix} \rightarrow \begin{pmatrix} x\\ y \end{pmatrix} \in \Gamma (T)$.
	
	Thus, $\Gamma(T)$ is closed.
	
	Now let $e_n = \{\delta_{kn}\}_k$ for $n\in\mathbb{N}$. Then 
	\begin{equation}
		||e_n|| = \sum_{k}|\delta_{kn}| = 1,
	\end{equation}
	so $e_n \in \ell^1$. Furthermore,
	\begin{equation}
		\sum_{k}k|\delta_{kn}| = n < \infty,
	\end{equation}
	so $e_n \in W$. However,
	\begin{equation}\label{delta_norm}
		||Te_n|| = ||\{k\delta_{kn}\}_k|| = \sum_{k}k|\delta_{kn}| = n.
	\end{equation}
	Since we can do this for any $n\in\mathbb{N}$, then $||T||\geq n \implies ||T||=\infty$.
	
	[\textit{Note:} It is somewhat tempting to take the result of equation \eqref{delta_norm} to suggest that since $n\in\mathbb{N}$ is finite, then $||T||<\infty$ so $T$ must be bounded. However, this is not necessarily true because even though each $\sum_k k |a_k|$ is finite, the supremum over all $||a||_1 = 1$ of the sum may not be. For instance, $\sup_{n\in \mathbb{N}} n = \infty$, even though each $n$ is finite.]
	
	Therefore we conclude that the graph of $T$ is closed, but $T$ is not bounded.
	
\end{proof}
\end{document}