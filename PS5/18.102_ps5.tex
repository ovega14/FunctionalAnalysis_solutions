\documentclass{article}
\usepackage[utf8]{inputenc}
\usepackage[english]{babel}
\usepackage[]{amsthm} %lets us use \begin{proof}
\usepackage[]{amssymb} %gives us the character \varnothing
\usepackage[]{amsmath}
\usepackage[parfill]{parskip} %avoid indent when skipping lines
\usepackage[toc,page]{appendix}
\usepackage{mathtools}
\usepackage{hyperref}
\hypersetup{
	colorlinks=true,
	linkcolor=blue,
	filecolor=magenta,      
	urlcolor=cyan,
	pdftitle={18.102-ps5},
	pdfpagemode=FullScreen,
}
%\urlstyle{same}
\newcommand{\R}{\mathbb{R}} %the real numbers
\newcommand{\N}{\mathbb{N}} %the natural numbers
\newcommand{\M}{\mathcal{M}} %set of all Lebesgue-measurable sets

\title{18.102 Assignment 5}
\author{Octavio Vega}
\date\today

\begin{document}
\maketitle

%%% Notation %%%
We denote by $\M$ the set of all Lebesgue-measurable subsets of $\R$.

\section*{Problem 1}
TODO TODO TODO
%%%%%%%%%%%%%%%%%%%%%%%%%%%%%%%%%%%%%%%%%%%%%%%%%%%%%%%%%%%%%%%%%%%%%%%%%%%%%%%%%%%%%%%%%%%%%%%%%%%%%%%%%%%%%%%%%%%%%%%%%%%%%%%%%%%%%%%%%%%%
\section*{Problem 2}
\subsection*{(a)}
\begin{proof}
	Let $a \in \R$.
	
	We can write
	\begin{equation}
		fg = \frac{1}{4}\left[(f+g)^2 - (f-g)^2\right].
	\end{equation}
	We showed in lecture 9 that linear combinations of measurable functions are measurable, so we need only show that $f^2$ and $g^2$ are measurable.
	
	\underline{Case 1}: $\alpha < 0$. Then,
	\begin{equation}
		(f^2)^{-1}\left((\alpha, \infty]\right) = (f^2)^{-1}\left([0,\infty]\right) = E \in \M.
	\end{equation}
	\underline{Case 2}: $\alpha \geq 0$. Then $\forall x \in E$,
	\begin{align}
		f^2(x) > \alpha \iff f(x) > \sqrt{\alpha} \textrm{ or } f(x) < -\sqrt{a} \\
		\implies (f^2)^{-1}\left((\alpha, \infty]\right) = [-\infty, -\sqrt{\alpha}) \cup (\sqrt{\alpha}, \infty] \in \M.
	\end{align}
	So $f^2$ is measurable, and by the same reasoning $g^2$ is measurable.
	
	Therefore, $fg$ is measurable.
\end{proof}

%%%%%%%%%%%%%%%%%%%%%%%%%%%%%%%%%%%%%%%%%%%%%%%%%%%%%%%%%%%%%%%%%%%%%%%%%%%%%%%%%%%%%%%%%%%%%%%%%%%%%%%%%%%%%%%%%%%%%%%%%%%%%%%%%%%%%%%%%%%%
\section*{Problem 3}
\subsection*{(a)}
\begin{proof}
	($\Rightarrow$) Suppose $f$ is measurable. 
	
	Let $\alpha \in \R$. We may express the preimage of the set $(\alpha, \infty]$ under the inverse of the restriction of $f$ to $E$ as follows:
	\begin{equation}\label{f-inv-E}
		f^{-1}\big|_E \left((\alpha, \infty])\right) = f^{-1}\left((\alpha, \infty])\right) \cap E,
	\end{equation}
	and similarly for $F$:
	\begin{equation}\label{f-inv-F}
		f^{-1}\big|_F \left((\alpha, \infty])\right) = f^{-1}\left((\alpha, \infty])\right) \cap F.
	\end{equation}
	Since $f$ is measurable, then $f^{-1}\left((\alpha, \infty])\right) \in \M$. By assumption, $E$ and $F$ are also measurable. Hence, the intersections in \eqref{f-inv-E} and \eqref{f-inv-F} are also measurable. 
	
	Therefore, $f\big|_E$ and $f\big|_F$ are measurable.
	
	($\Leftarrow$) Suppose $f\big|_E$ and $f\big|_F$ are measurable.
	
	Then for ever $\alpha \in \R$, $f^{-1}\big|_E \left((\alpha, \infty])\right) \in \M$ and $f^{-1}\big|_F \left((\alpha, \infty])\right) \in \M$. Since $\M$ is closed under taking finite unions, then the union of each of these sets is also measurable, i.e.
	\begin{equation}
		f^{-1}\big|_E \left((\alpha, \infty])\right) \cup f^{-1}\big|_F \left((\alpha, \infty])\right) \in \M.
	\end{equation}
	We also have that $E, F \in \M$, so $E \cup F \in \M$. Then we have
	\begin{align}
		f^{-1}\big|_E \left((\alpha, \infty])\right) \cup f^{-1}\big|_F \left((\alpha, \infty])\right) \\= \left(f^{-1}\left((\alpha, \infty])\right) \cap E\right) \cup \left(f^{-1}\left((\alpha, \infty])\right) \cap F\right) \nonumber \\
		= f^{-1}\left((\alpha,\infty]\right) \cap (E \cup F) \\ 
		= f^{-1}\left((\alpha,\infty]\right) \in \M, \label{inv-subset-union}
	\end{align}
	where in line \eqref{inv-subset-union} we used the fact that $f^{-1}\left((\alpha,\infty]\right) \subset (E \cup F)$. 
	
	Therefore, as desired, $f$ must be measurable.
\end{proof}

\subsection*{(b)}
\begin{proof}
	($\Rightarrow$) Suppose $f$ is measurable.
	
	We define the indicator function $\chi_E$ on $E$ via
	\begin{equation}
		\chi_E(x) := 
		\begin{cases}
			1, \; x \in E \\
			0, \; x \in E^c.
		\end{cases}
	\end{equation}
	Then we can express $g$ as the product
	\begin{equation}
		g(x) = f(x) \cdot \chi_E(x).
	\end{equation}
	In problem \textbf{2a}, we showed that the product of measurable functions is measurable. By assumption, $f$ is measurable. so we need only check that $\chi_E$ is measurable.
	
	Let $\alpha \in \R$. 
	
	\underline{Case 1}: $1 \leq \alpha \leq \infty$. Then,
	\begin{equation}
		\chi_E ^{-1}\left((\alpha, \infty]\right) = \emptyset \in \M.
	\end{equation}
	\underline{Case 2}: $0 \leq \alpha < 1$. Then,
	\begin{equation}
		\chi_E^{-1}\left((\alpha,\infty]\right) = E \in \M.
	\end{equation}
	\underline{Case 3}: $-\infty \leq \alpha < 0$. Then,
	\begin{equation}
		\chi_E^{-1}\left((\alpha, \infty]\right) = \R \in \M.
	\end{equation}
	 Hence, $\chi_E$ is measurable, so $f\cdot \chi_E$ is also measurable.
	 
	 Therefore, $g$ is measurable.
	 
	 ($\Leftarrow$) Suppose $g$ is measurable. Since $g: E\cup E^c = \R \rightarrow [-\infty, \infty]$ is defined by
	 \begin{align}
	 	g(x) := 
	 	\begin{cases}
	 		f(x), \; &x \in E \\
	 		0, \; &x \in E^c,
	 	\end{cases}
	 \end{align}
	 then by restricting $g$ to $E$ we get $g\big|_E(x) = f(x)$. By part \textbf{(a)}, $g\big|_E$ must be measurable.
	 
	 Therefore, $f$ is measurable.
\end{proof}

\subsection*{(c)}
\begin{proof}
	We have already shown in class that sums and products of measurable functions are measurable. So if $u$ and $v$ are measurable, then both $u^2$ and $v^2$ are measurable, which implies that $u^2 + v^2$ is measurable.
	
	Define
	\begin{equation}
		f(x) := u^2(x) + v^2(x).
	\end{equation}
	Then we need only check that $f^{\frac{1}{2}}$ is measurable.
	
	Let $g(x) = x^{\frac{1}{2}}$. Then $f^{\frac{1}{2}}(x) = (g \circ f)(x)$, and $f: E \rightarrow [0,\infty] \\ \implies g: [0, \infty] \rightarrow [0, \infty]$. We use the fact that the composition of measurable functions is measurable, proven in appendix \ref{appendix:A.1}, to show that $f^{\frac{1}{2}}$ is measurable. 
	
	Let $\alpha \in \R$.
	
	\underline{Case 1}: $0 \leq \alpha \leq \infty$. Then,
	\begin{equation}
		g^{-1}\left((\alpha, \infty]\right) = (\alpha^2, \infty) \in \M.
	\end{equation}
	\underline{Case 2}: $-\infty \leq \alpha < 0$. Then,
	\begin{equation}
		g^{-1}\left((\alpha, \infty]\right) = \emptyset \in \M.
	\end{equation}
	Hence, $g$ is measurable, so by \ref{appendix:A.1}, $f^{\frac{1}{2}}$ is measurable.
	
	Therefore, $(u^2 + v^2)^{\frac{1}{2}}$ is measurable.
\end{proof}
%%%%%%%%%%%%%%%%%%%%%%%%%%%%%%%%%%%%%%%%%%%%%%%%%%%%%%%%%%%%%%%%%%%%%%%%%%%%%%%%%%%%%%%%%%%%%%%%%%%%%%%%%%%%%%%%%%%%%%%%%%%%%%%%%%%%%%%%%%%%
\section*{Problem 4}
TODO TODO TODO

%%%%%%%%%%%%%%%%%%%%%%%%%%%%%%%%%%%%%%%%%%%%%%%%%%%%%%%%%%%%%%%%%%%%%%%%%%%%%%%%%%%%%%%%%%%%%%%%%%%%%%%%%%%%%%%%%%%%%%%%%%%%%%%%%%%%%%%%%%%%
%%%%%%%%%%%%%%%%%%%%%%%%%%%%%%%%%%%%%%%%%%%%%%%%%%%%%%%%%%%%%%%%%%%%%%%%%%%%%%%%%%%%%%%%%%%%%%%%%%%%%%%%%%%%%%%%%%%%%%%%%%%%%%%%%%%%%%%%%%%%
\begin{appendices}
	
\section{Appendix A}
\subsection{}
\label{appendix:A.1}
TODO TODO TODO

\end{appendices}

\end{document}