\documentclass{article}
\usepackage[utf8]{inputenc}
\usepackage[english]{babel}
\usepackage[]{amsthm} %lets us use \begin{proof}
\usepackage[]{amssymb} %gives us the character \varnothing
\usepackage[]{amsmath}
\usepackage[parfill]{parskip} %avoid indent when skipping lines
\usepackage[toc,page]{appendix}
\usepackage{mathtools}
\usepackage{hyperref}
\hypersetup{
	colorlinks=true,
	linkcolor=blue,
	filecolor=magenta,      
	urlcolor=cyan,
	pdftitle={18.102-ps5},
	pdfpagemode=FullScreen,
}
%\urlstyle{same}
\newcommand{\R}{\mathbb{R}} %the real numbers
\newcommand{\N}{\mathbb{N}} %the natural numbers
\newcommand{\M}{\mathcal{M}} %set of all Lebesgue-measurable sets
\newcommand{\Q}{\mathbb{Q}} % the rational numbers

\title{18.102 Assignment 5}
\author{Octavio Vega}
\date\today

\begin{document}
\maketitle

%%% Notation %%%
In the problems to follow, we denote by $\M$ the set of all Lebesgue-measurable subsets of $\R$.

\section*{Problem 1}
\subsection*{(a)}
\begin{proof}
	Since $E, M \in \M$, then $E \cup M \in \M$ and $E \cap M \in \M$. We can express the union as 
	\begin{equation}
		E \cup F = (E \cap F) \cup (E \backslash F) \cup (F \backslash E).
	\end{equation}
	Then since $E$ and $F \backslash E$ are disjoint, we have
	\begin{align}
		m(E \cup F) + m(E \cap F) &= m\left[(E \cap F) \cup (E \backslash F)\right] + m(F \backslash E) + m(E \cap F) \\
		&= m\left[(E \cap F) \cup (E \backslash F)\right] + m\left[(F \backslash E) \cup (E \cap F)\right] \\
		&= m(E) + m(F),
	\end{align}
	as desired.	
\end{proof}

\subsection*{(b)}
\begin{proof}
	Since $m(E_1) < \infty$ and $E_{k+1} \subset E_k$ $\forall k \in \N$, then $m(E_k) < m(E_1) < \infty$. Then
	\begin{equation}
		m\left(\bigcap_{k=1}^{\infty} E_k \right) < \infty,
	\end{equation}
	since $\bigcap_{k=1}^{\infty} E_k \subset E_k$.
	
	Additionally, we have
	\begin{equation}\label{1b-star}
		m\left[E_1 \backslash \left(\bigcap_{k=1}^{\infty}E_k\right)\right] = m(E_1) - m\left(\cap_{k=1}^{\infty}E_k\right).
	\end{equation}
	Also, we note that
	\begin{align}
		E_1 \backslash \left(\bigcap_{k=1}^{\infty} E_k \right) &= E_1 \cap \left(\bigcap_{k=1}^{\infty}E_k\right)^c \\
		&= E_1 \cap \left(\bigcup_{k=1}^{\infty} E_k^c\right) \\
		&= \bigcup_{k=1}^{\infty} \left(E_1 \cap E_k^c\right) \\
		&= \bigcup_{k=1}^{\infty}\left(E_1 \backslash E_k \right). \label{1b-star-star}
	\end{align}
	Define $U_k = E_1 \backslash E_k$ for each $k \in \N$. Then $\forall k \in \N$, $U_k \subset U_{k+1}$ since $E_{k+1} \subset E_k$. By \eqref{1b-star-star}, we have
	\begin{align}
		m\left(E_1 \backslash \bigcap_{k=1}^{\infty}E_k\right) &= m\left[\bigcup_{k=1}^{\infty}(E_1 \backslash E_k)\right] \\
		&= \lim_{k \to \infty} m(E_1 \backslash E_k) \label{1b-appendix}\\
		&= \lim_{k \to \infty} \left[m(E_1) - m(E_k)\right] \\
		&= m(E_1) - \lim_{k \to \infty}m(E_k),
	\end{align}
	where we get to line \eqref{1b-appendix} using the theorem proven in appendix \ref{appendix:A}. Then using equation \eqref{1b-star}, this gives
	\begin{align}
		m(E_1) - \lim_{k \to \infty} m(E_k) = m(E_1) - m\left(\bigcap_{k=1}^{\infty} E_k \right).
	\end{align}
	Therefore, we conclude that $m\left(\bigcap_{k=1}^{\infty}E_k\right) = \lim_{k \to \infty} m(E_k)$.\footnote{The assignment states this result as \textit{continuity from below} for the Lebesgue measure, but I believe this is actually \textit{continuity from above}.}
\end{proof}
%%%%%%%%%%%%%%%%%%%%%%%%%%%%%%%%%%%%%%%%%%%%%%%%%%%%%%%%%%%%%%%%%%%%%%%%%%%%%%%%%%%%%%%%%%%%%%%%%%%%%%%%%%%%%%%%%%%%%%%%%%%%%%%%%%%%%%%%%%%%
\section*{Problem 2}
\subsection*{(a)}
\begin{proof}
	Let $a \in \R$.
	
	We can write
	\begin{equation}
		fg = \frac{1}{4}\left[(f+g)^2 - (f-g)^2\right].
	\end{equation}
	We showed in lecture 9 that linear combinations of measurable functions are measurable, so we need only show that $f^2$ and $g^2$ are measurable.
	
	\underline{Case 1}: $\alpha < 0$. Then,
	\begin{equation}
		(f^2)^{-1}\left((\alpha, \infty]\right) = (f^2)^{-1}\left([0,\infty]\right) = E \in \M.
	\end{equation}
	\underline{Case 2}: $\alpha \geq 0$. Then $\forall x \in E$,
	\begin{align}
		f^2(x) > \alpha \iff f(x) > \sqrt{\alpha} \textrm{ or } f(x) < -\sqrt{a} \\
		\implies (f^2)^{-1}\left((\alpha, \infty]\right) = [-\infty, -\sqrt{\alpha}) \cup (\sqrt{\alpha}, \infty] \in \M.
	\end{align}
	So $f^2$ is measurable, and by the same reasoning $g^2$ is measurable.
	
	Therefore, $fg$ is measurable.
\end{proof}

\subsection*{(b)}
\begin{proof}
	Let $\alpha \in \R$.
	
	\underline{Case 1}: $\alpha = +\infty$. Then,
	\begin{equation}
		h^{-1}(\{\infty\}) = \{a\} = \bigcap_{n=1}^{\infty}\left[a, a + \frac{1}{n}\right] \in \M.
	\end{equation}
	\underline{Case 2}: $\alpha \neq \infty$. We have
	\begin{equation}
		h^{-1}\left((\alpha, \infty]\right) = (f+g)^{-1}\left((\alpha, \infty]\right).
	\end{equation}
	Then $x \in (f+g)^{-1}\left((\alpha, \infty]\right) \iff f(x) + g(x) > \alpha$. By the density of $\Q$ in $\R$, $\exists r \in \Q$ such that $f(x) > r > \alpha - g(x)$. Then since $f$ and $g$ are measurable, we have
	\begin{equation}
		(f+g)^{-1}\left((\alpha, \infty]\right) = \bigcup_{r \in \Q} \left[f^{-1}\left((r,\infty]\right) \cap g^{-1}\left((\alpha - r,\infty]\right)\right] \in \M.
	\end{equation}
	Therefore, $h$ is measurable.
\end{proof}
%%%%%%%%%%%%%%%%%%%%%%%%%%%%%%%%%%%%%%%%%%%%%%%%%%%%%%%%%%%%%%%%%%%%%%%%%%%%%%%%%%%%%%%%%%%%%%%%%%%%%%%%%%%%%%%%%%%%%%%%%%%%%%%%%%%%%%%%%%%%
\section*{Problem 3}
\subsection*{(a)}
\begin{proof}
	($\Rightarrow$) Suppose $f$ is measurable. 
	
	Let $\alpha \in \R$. We may express the preimage of the set $(\alpha, \infty]$ under the inverse of the restriction of $f$ to $E$ as follows:
	\begin{equation}\label{f-inv-E}
		f^{-1}\big|_E \left((\alpha, \infty])\right) = f^{-1}\left((\alpha, \infty])\right) \cap E,
	\end{equation}
	and similarly for $F$:
	\begin{equation}\label{f-inv-F}
		f^{-1}\big|_F \left((\alpha, \infty])\right) = f^{-1}\left((\alpha, \infty])\right) \cap F.
	\end{equation}
	Since $f$ is measurable, then $f^{-1}\left((\alpha, \infty])\right) \in \M$. By assumption, $E$ and $F$ are also measurable. Hence, the intersections in \eqref{f-inv-E} and \eqref{f-inv-F} are also measurable. 
	
	Therefore, $f\big|_E$ and $f\big|_F$ are measurable.
	
	($\Leftarrow$) Suppose $f\big|_E$ and $f\big|_F$ are measurable.
	
	Then for ever $\alpha \in \R$, $f^{-1}\big|_E \left((\alpha, \infty])\right) \in \M$ and $f^{-1}\big|_F \left((\alpha, \infty])\right) \in \M$. Since $\M$ is closed under taking finite unions, then the union of each of these sets is also measurable, i.e.
	\begin{equation}
		f^{-1}\big|_E \left((\alpha, \infty])\right) \cup f^{-1}\big|_F \left((\alpha, \infty])\right) \in \M.
	\end{equation}
	We also have that $E, F \in \M$, so $E \cup F \in \M$. Then we have
	\begin{align}
		f^{-1}\big|_E \left((\alpha, \infty])\right) \cup f^{-1}\big|_F \left((\alpha, \infty])\right) \\= \left(f^{-1}\left((\alpha, \infty])\right) \cap E\right) \cup \left(f^{-1}\left((\alpha, \infty])\right) \cap F\right) \nonumber \\
		= f^{-1}\left((\alpha,\infty]\right) \cap (E \cup F) \\ 
		= f^{-1}\left((\alpha,\infty]\right) \in \M, \label{inv-subset-union}
	\end{align}
	where in line \eqref{inv-subset-union} we used the fact that $f^{-1}\left((\alpha,\infty]\right) \subset (E \cup F)$. 
	
	Therefore, as desired, $f$ must be measurable.
\end{proof}

\subsection*{(b)}
\begin{proof}
	($\Rightarrow$) Suppose $f$ is measurable.
	
	We define the indicator function $\chi_E$ on $E$ via
	\begin{equation}
		\chi_E(x) := 
		\begin{cases}
			1, \; x \in E \\
			0, \; x \in E^c.
		\end{cases}
	\end{equation}
	Then we can express $g$ as the product
	\begin{equation}
		g(x) = f(x) \cdot \chi_E(x).
	\end{equation}
	In problem \textbf{2a}, we showed that the product of measurable functions is measurable. By assumption, $f$ is measurable. so we need only check that $\chi_E$ is measurable.
	
	Let $\alpha \in \R$. 
	
	\underline{Case 1}: $1 \leq \alpha \leq \infty$. Then,
	\begin{equation}
		\chi_E ^{-1}\left((\alpha, \infty]\right) = \emptyset \in \M.
	\end{equation}
	\underline{Case 2}: $0 \leq \alpha < 1$. Then,
	\begin{equation}
		\chi_E^{-1}\left((\alpha,\infty]\right) = E \in \M.
	\end{equation}
	\underline{Case 3}: $-\infty \leq \alpha < 0$. Then,
	\begin{equation}
		\chi_E^{-1}\left((\alpha, \infty]\right) = \R \in \M.
	\end{equation}
	 Hence, $\chi_E$ is measurable, so $f\cdot \chi_E$ is also measurable.
	 
	 Therefore, $g$ is measurable.
	 
	 ($\Leftarrow$) Suppose $g$ is measurable. Since $g: E\cup E^c = \R \rightarrow [-\infty, \infty]$ is defined by
	 \begin{align}
	 	g(x) := 
	 	\begin{cases}
	 		f(x), \; &x \in E \\
	 		0, \; &x \in E^c,
	 	\end{cases}
	 \end{align}
	 then by restricting $g$ to $E$ we get $g\big|_E(x) = f(x)$. By part \textbf{(a)}, $g\big|_E$ must be measurable.
	 
	 Therefore, $f$ is measurable.
\end{proof}

\subsection*{(c)}
\begin{proof}
	We have already shown in class that sums and products of measurable functions are measurable. So if $u$ and $v$ are measurable, then both $u^2$ and $v^2$ are measurable, which implies that $u^2 + v^2$ is measurable.
	
	Define
	\begin{equation}
		f(x) := u^2(x) + v^2(x).
	\end{equation}
	Then we need only check that $f^{\frac{1}{2}}$ is measurable.
	
	Let $g(x) = x^{\frac{1}{2}}$. Then $f^{\frac{1}{2}}(x) = (g \circ f)(x)$, and $f: E \rightarrow [0,\infty] \\ \implies g: [0, \infty] \rightarrow [0, \infty]$. We use the fact that the composition of a continuous function ($g$) with a measurable function ($f$) is measurable to show that $f^{\frac{1}{2}}$ is measurable. 
	
	Let $\alpha \in \R$.
	
	\underline{Case 1}: $0 \leq \alpha \leq \infty$. Then,
	\begin{equation}
		g^{-1}\left((\alpha, \infty]\right) = (\alpha^2, \infty) \in \M.
	\end{equation}
	\underline{Case 2}: $-\infty \leq \alpha < 0$. Then,
	\begin{equation}
		g^{-1}\left((\alpha, \infty]\right) = \emptyset \in \M.
	\end{equation}
	Hence, $g$ is measurable, so $f^{\frac{1}{2}}$ is measurable.
	
	Therefore, $(u^2 + v^2)^{\frac{1}{2}}$ is measurable.
\end{proof}
%%%%%%%%%%%%%%%%%%%%%%%%%%%%%%%%%%%%%%%%%%%%%%%%%%%%%%%%%%%%%%%%%%%%%%%%%%%%%%%%%%%%%%%%%%%%%%%%%%%%%%%%%%%%%%%%%%%%%%%%%%%%%%%%%%%%%%%%%%%%
\section*{Problem 4}
\subsection*{(a)}
\begin{proof}
	We define $S$ as the set of elements in $E$ for which $f_n$ does not converge to $f$, via 
	\begin{equation}
		S := \{x\in E \;|\; \lim_{n \to \infty}f_n(x) \neq f(x) \}.
	\end{equation}
	If $\lim_{n \to \infty} f_n(x) \neq f(x)$, then for every $n \in \N$, $\exists \epsilon_0$ such that $\forall m > n$, \\$|f_m(x) - f(x)| \geq \epsilon_0$. Since this statement holds for every $n$, then letting $\epsilon_0 = \frac{1}{k}$ allows us to express $S$ as a countable union:
	\begin{equation}
		S = \bigcup_{m=1}^{\infty} \{x \in E \;|\; |f_m(x) - f(x)| \geq k^{-1} \} = F_1(k).
	\end{equation}
	But since $f_n \rightarrow f$ almost everywhere (a.e.) on $E$, then $S$ must have measure $0$, i.e.
	\begin{equation}\label{4a-star}
		m(F_1(k)) = 0 < \infty.
	\end{equation}
	We can also check that
	\begin{align}
		F_{n+1}(k) &= \bigcup_{m=n+1}^{\infty}\{x \in E \;|\; |f_m(x) - f(x)| \geq k^{-1} \} \\
		&\subset \bigcup_{m=n}^{\infty}\{x \in E \;|\; |f_m(x) - f(x)| \geq k^{-1}\} \\
		&= F_n(k).
	\end{align}
	So $\forall n \in \N$,
	\begin{equation}\label{4a-star-star}
		F_{n+1}(k) \subset F_n(k).
	\end{equation}
	By equations \eqref{4a-star} and \eqref{4a-star-star}, we appeal to problem \textbf{1b} to deduce that
	\begin{equation}
		\lim_{n \to \infty} m(F_n(k)) = m\left(\bigcap_{n=1}^{\infty} F_n(k) \right).
	\end{equation}
	We also have that
	\begin{equation}
		\bigcap_{n=1}^{\infty} F_n(k) \subset F_n(k) \subset F_1(k),
	\end{equation}
	so by the monotonicity of the Lebesgue measure, 
	\begin{equation}
		m\left(\bigcap_{n=1}^{\infty}F_n(k)\right) \leq m(F_1(k)) = 0.
	\end{equation}
	Therefore, $\lim_{n \to \infty} m(F_n(k)) = 0$.
\end{proof}

\subsection*{(b)}
\begin{proof}
	(i) Since $F$ is a countable union, then
	\begin{align}
		m(F) &= m\left(\bigcup_{k=1}^{\infty} F_{n_k}(k)\right) \\
		& \leq \sum_{k=1}^{\infty} m\left(F_{n_k}(k)\right) \\
		&< \sum_{k=1}^{\infty}2^{-k}\epsilon \\
		&= \epsilon\left(\frac{1}{1-\frac{1}{2}} - 1\right) \\
		&= \epsilon.
	\end{align}
	Thus, $m(F) < \epsilon$.
	
	(ii) The complement $F^c$ is given by
	\begin{equation}
		F^c = \left(\bigcup_{k=1}^{\infty} F_{n_k}(k)\right)^c = \bigcap_{k=1}^{\infty} F_{n_k}(k)^c,
	\end{equation}
	where
	\begin{equation}
		F_{n_k}(k) := \bigcup_{m=n_k}^{\infty} \{x \in E \;|\; |f_m(x) - f(x)|\geq k^{-1}\}.
	\end{equation}
	Then,
	\begin{align}
		F_{n_k}(k)^c = \bigcap_{m=n_k}^{\infty}\{x \in E \;|\; |f_n(x) - f(x)| < k^{-1}\} \\
		\implies F^c = \bigcap_{k=1}^{\infty}\bigcap_{m=n_k}^{\infty}\{x \in E \;|\; |f_n(x) - f(x)| < k^{-1}\}.
	\end{align}
	Let $x \in F^c$. Then by the above equation, $\forall k \in \N$, $\exists n_k \in \N$ such that $\forall m \geq n_k$, $|f_m(x) - f(x)| < k^{-1}$. This implies that $f_n(x) \rightarrow f(x)$, and this is true for every $x \in F^c$.
	
	Therefore, we conclude that $f_n \rightarrow f$ uniformly on $F^c$.
\end{proof}
\underline{Remark}: In this problem, we have effectively proven \textit{Littlewood's second principle}, which is that every convergent sequence of measurable functions is uniformly convergent.
%%%%%%%%%%%%%%%%%%%%%%%%%%%%%%%%%%%%%%%%%%%%%%%%%%%%%%%%%%%%%%%%%%%%%%%%%%%%%%%%%%%%%%%%%%%%%%%%%%%%%%%%%%%%%%%%%%%%%%%%%%%%%%%%%%%%%%%%%%%%
%%%%%%%%%%%%%%%%%%%%%%%%%%%%%%%%%%%%%%%%%%%%%%%%%%%%%%%%%%%%%%%%%%%%%%%%%%%%%%%%%%%%%%%%%%%%%%%%%%%%%%%%%%%%%%%%%%%%%%%%%%%%%%%%%%%%%%%%%%%%
\begin{appendices}
	
\section{Appendix A}
\label{appendix:A}
\textbf{Theorem} (\textit{Continuity from Below}): If $\{E_n\}_{n=1}^{\infty}$ is a countable collection of measurable sets such that $\forall n \in \N$, $E_n \subset E_{n+1}$ and $m(E_n) < \infty$, then \\$m\left(\bigcup_{n=1}^{\infty}E_n\right) = \lim_{n \to \infty} m(E_n)$.
\begin{proof}
	We can write the countable union of all the sets as
	\begin{align}
		\bigcup_{n=1}^{\infty}E_n &= E_1 \cup (E_2 \backslash E_1) \cup (E_3 \backslash E_2) \cup \cdots \\
		&= E_1 \cup \left(\bigcup_{n=2}^{\infty} (E_n\backslash E_{n-1})\right). \label{app-A-1}
	\end{align}
	Since $E_n \subset E_{n+1}$ $\forall n \in \N$, then $E_1$ and $\{E_n \backslash E_{n-1}\}_{n \geq 2}$ are all disjoint. Then taking the Lebesgue measure on both sides of \eqref{app-A-1} yields
	\begin{align}
		m\left(\bigcup_{n=1}^{\infty} E_n \right) &= m\left(E_1 \cup \left(\bigcup_{n=2}^{\infty}(E_n \backslash E_{n-1})\right)\right) \\
		&= m(E_1) + \sum_{n=2}^{\infty}m(E_n \backslash E_{n-1}) \\
		&= m(E_1) + \sum_{n=2}^{\infty}\left[m(E_n) - m(E_{n-1})\right] \\
		&= m(E_1) + \lim_{n \to \infty}\sum_{k=2}^{n}\left[m(E_k) - m(E_{k-1})\right] \\
		&= m(E_1) + \lim_{n \to \infty}\left(m(E_n) - m(E_1)\right) \\
		&= \lim_{n \to \infty} m(E_n),
	\end{align}
	as desired.
\end{proof}

\end{appendices}

\end{document}