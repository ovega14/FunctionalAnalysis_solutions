\documentclass{article}
\usepackage[utf8]{inputenc}
\usepackage[english]{babel}
\usepackage[]{amsthm} %lets us use \begin{proof}
\usepackage[]{amssymb} %gives us the character \varnothing
\usepackage[]{amsmath}
\usepackage[parfill]{parskip} %avoid indent when skipping lines

\title{18.102 Assignment 3}
\author{Octavio Vega}
\date\today

\begin{document}
\maketitle

\section*{Problem 1}
\subsection*{(a)}
\begin{proof}
	We want to show that $u \in M'$. First, we show that $u$ is linear.
	
	Let $a,b \in M$ and let $\lambda \in \mathbb{C}$. Then
	\begin{equation}
		u(\lambda a ) = \lim_{k\to\infty}(\lambda a_k) = \lambda \cdot \lim_{k \to \infty}a_k = \lambda u(a) \textrm{ , and}
	\end{equation} 
	\begin{equation}
		u(a+b) = \lim_{k\to\infty}(a_k + b_k) = \lim_{k \to \infty} a_k + \lim_{k \to \infty}b_k = u(a) + u(b).
	\end{equation}
	So $u$ is linear on $M$. Next, we show that $u$ is bounded.
	
	Let $a\in M$, i.e. $\lim_{k \to \infty} a_k$ exists. Then $a$ is bounded, so $\exists B \geq 0$ such that $\forall k\in\mathbb{N}$, $|a_k| \leq B$. Then by continuity of the norm,
	\begin{align}
		||u|| &\leq |u(a)|  \\
		&= \left|\lim_{k \to \infty}a_k\right| \\
		&= \lim_{k \to \infty} |a_k| \\
		&\leq B,
	\end{align} 
	so $u$ is bounded.
	
	Then we conclude that $u$ is a bounded linear functional on $M$. 
\end{proof}

\subsection*{(b)}
\begin{proof}
	(By contradiction). Suppose instead that $\exists b \in \ell^1$ such that $\forall a \in \ell^{\infty}$,
	\begin{equation}\label{sum-operator}
		v(a) = \sum_{k=1}^{\infty}a_k b_k.
	\end{equation}
	Define $e_n := \{\delta_{kn}\}_k \in \ell^{\infty}$, for fixed $n\in\mathbb{N}$. Then $\lim_{k \to \infty}\delta_{nk} = 0$, so $e_n \in M$ as well. By equation \eqref{sum-operator}, we have
	\begin{equation}
		v(e_n) = \sum_{k=1}^{\infty}\delta_{kn}b_k = b_n.
	\end{equation}
	By the Hahn-Banach theorem, $v|_{M} = u$. But $u(e_n) = \lim_{k \to \infty}\delta_{kn}=0$, and since $e_n \in M$, we have
	\begin{equation}
		b_n = v(e_n)=u(e_n) = 0.
	\end{equation}
	This must hold for any $n\in\mathbb{N}$, so $b_n=0$ $\forall n\in\mathbb{N}$. Then $b=\{b_k\}_k = (0, 0, ...)$, so $v=0$ by definition. But
	\begin{equation}
		0 = v(1, 1, ...) = u(1, 1, ....) = 1, \quad (\Rightarrow \Leftarrow)
	\end{equation}
	so we arrive at a contradiction to the initial assumption. 
	
	Therefore $\nexists b \in \ell^1$ such that $\forall a \in \ell^{\infty}$, $v(a)=\sum_k a_k b_k$.
\end{proof}
%%%%%%%%%%%%%%%%%%%%%%%%%%%%%%%%%%%%%%%%%%%%%%%%%%%%%%%%%%%%%%%%%%%%%%%%%%%%%%%%%%%%%%%%%%%%%%%%%%%%%%%%%%%%%%%%%%%%%%%%%%%%%%%%%%%%%%%%%%%
\section*{Problem 2}
\subsection*{(a)}
\begin{proof}
	First we show that $\|T^{\dagger}\| \leq \|T\|$. We have\begin{align}
		\|T^{\dagger}\| &= \sup_{||f||=1}\|T^{\dagger} f\| \\
		&= \sup_{\|f\|=1}\|f\circ T\| \\
		&\leq \sup_{\|f\|=1}\|f\|\|T\| \\
		&\leq \|T\|.
	\end{align}
	So, $T^{\dagger}: W' \rightarrow V'$ is bounded, i.e. $T^{\dagger}\in \mathcal{B}(W', V')$. 
	
	Next we show that $\|T^{\dagger}\| \geq \|T\|$. 
	
	Let $x\in V$ with $\|x\|=1$. Since $W$ is a normed space, then by the theorem from lecture 6 (corollary to Hahn-Banach Thm.), $\exists f \in W'$ such that $\|f\|=1$ and $f(w)=\|w\|$ $\forall w \in W\backslash\{0\}$. 
	
	Since $T: V \rightarrow W$, then $w = Tx$ for $x \in V$, so $f(Tx) = \|Tx\|$. Then we have
	\begin{align}
		\|T^{\dagger}\| &= \sup_{\|f\|=1}\|T^{\dagger}f\| \\
		&= \sup_{\|f\|=1}\sup_{\|x\|=1}|(f\circ T)(x)| \\
		&=\sup_{\|f\|=1}\sup_{\|x\|=1}|f(Tx)| \\
		&\geq \sup_{\|x\|=1}\|Tx\| \\
		&= \|T\|,
	\end{align}
	as desired.
	
	Thus $T^{\dagger}\in\mathcal{B}(W', V')$ and $\|T^{\dagger}\| = \|T\|$.
\end{proof}

\subsection*{(b)}
\begin{proof}
	Let $a \in \ell^p$, and $b_k=R a_k$. We want to show that $b = \{b_k\}_k \in \ell^p$. Since $a \in \ell^p$, then $a$ is bounded, which implies that $\exists B \geq 0$ such that $\forall k \in \mathbb{N}$, $|a_k| \leq B$. Then $b_k = Ra_k := a_{k-1}$, with $b_1 = a_0 := 0$.
	
	Thus $\forall k \in \mathbb{N}$, $|b_k| = |a_{k-1}|\leq B$, so $b$ is also bounded. We have
	\begin{equation}
		\|b\|_{\infty} = \sup_k |b_k| = \sup_k |a_{k-1}|.
	\end{equation}
	So, $R: \ell^p \rightarrow \ell^p$. Next we compute the operator norm of $R$:
	\begin{align}
		\|R\| &= \sup_{\|a\|=1}\|Ra\| \\
		&= \sup_{\|a\|=1}\sup_k |Ra_k| \\
		&= \sup_{\|a\|=1}\sup_k|a_{k-1}| \\
		&= \sup_{\|a\|=1}\sup_k \{0, |a_1|, |a_2|, ...\} \\
		&= \sup_{\|a\|=1}\sup_k |a_k| \\
		&= \sup_{\|a\|=1}\|a\| \\
		&= 1.
	\end{align}
	Therefore, $R\in\mathcal{B}(\ell^p, \ell^p)$ with $\|R\|=1$.
\end{proof}

\subsection*{(c)}
Suppose $1\leq p < \infty$, and $\frac{1}{p} + \frac{1}{q} = 1$. From assignment 1, we identify $\left(\ell^p\right)'$ with $\ell^q$ via the pairing: $f\in \left(\ell^p\right)' \iff \exists b\in \ell^q$ such that $\forall a \in \ell^p$, 
\begin{equation}
	f(a) = \sum_{k=1}^{\infty} a_k b_k,
\end{equation}
and $\|f\| = \|b\|_q$. 

For example, let $b=\{b_k\}_k=e_1$ defined by $e_1 := \{\delta_{1k}\}_k=\{1, 0, 0, ...\} \in \ell^q$. Then $\forall a \in ell^p$,
\begin{align}
	\left(R^{\dagger}e_1\right)(a) &:= \sum_{k=1}^{\infty}Ra_k \delta_{1k} \\
	&= Ra_1\\
	&= a_0  \\
	&= 0 \\
	&= \sum_{k=1}^{\infty}a_k \cdot 0.
\end{align}
Thus $R^{\dagger}e_1 = 0 \in \ell^q$. 

Now let $a\in\ell^p$. Then 
\begin{align}
	\left(R^{\dagger}b\right)(a) &= \sum_{k=1}^{\infty}\left(Ra\right)_k b_k \\
	&= \sum_{k=1}^{\infty} a_{k-1}b_k \\
	&= 0\cdot b_k + \sum_{k=2}^{\infty}a_{k-1}b_k \\
	&= \sum_{k=1}^{\infty}a_k b_{k+1}. 
\end{align}
Hence, $\{\left(R^{\dagger}b\right)_k\}_k = \{b_{k+1}\}_k$. Therefore, where $R$ was the right-shift operator, we can identify $R^{\dagger}$ as a left-shift operator.
%%%%%%%%%%%%%%%%%%%%%%%%%%%%%%%%%%%%%%%%%%%%%%%%%%%%%%%%%%%%%%%%%%%%%%%%%%%%%%%%%%%%%%%%%%%%%%%%%%%%%%%%%%%%%%%%%%%%%%%%%%%%%%%%%%%%%%%%%%%
\section*{Problem 3}
\begin{proof}
	To show that $m^*(E+x)=m^*(E)$, we will show both that \\$m^*(E+x)\geq m^*(E)$ and $m^*(E+x)\leq m^*(E)$.
	
	Let $\{I_n\}_{n\in\mathbb{N}}$ be a sequence of open intervals that covers $E$. Then $\{I_n + x\}_n$ covers $E+x$. Since interval length is invariant under translation, we have
	\begin{equation}
		m^*(E+x) \leq \sum_n \ell(I_n+x) = \sum_n\ell(I_n).
	\end{equation}
	So for every such sequence of intervals $\{I_n\}_n$, we have $m^*(E+x)\leq \sum_n\ell(I_n)$. Thus, 
	\begin{equation}\label{measure_leq}
		m^*(E+x)\leq m^*(E).
	\end{equation}
	Now let $\{I_n\}_{n\in\mathbb{N}}$ be a sequence of open intervals that covers $E+x$. Then $\{I_n-x\}_n$ covers $E$. Once again, by the translation-invariance of interval length, we have
	\begin{equation}
		m^*(E) \leq \sum_n\ell(I_n-x) = \sum_n\ell(I_n).
	\end{equation}
	Since this is true for every such sequence of intervals $\{I_n\}$, then
	\begin{equation}\label{measure_geq}
		m^*(E) \leq m^*(E+x).
	\end{equation}
	From equations \eqref{measure_leq} and \eqref{measure_geq}, we conclude that $m^*(E+x)=m^*(E)$.
	
	Therefore, the outer measure $m^*$ is translation-invariant.
\end{proof}
%%%%%%%%%%%%%%%%%%%%%%%%%%%%%%%%%%%%%%%%%%%%%%%%%%%%%%%%%%%%%%%%%%%%%%%%%%%%%%%%%%%%%%%%%%%%%%%%%%%%%%%%%%%%%%%%%%%%%%%%%%%%%%%%%%%%%%%%%%%
\section*{Problem 4}
\subsection*{(a)}
\begin{proof}
	Let $x \in U$. Then $a_x \leq b_x$, and both $(a_x, x]\subset U$ and $[x, b_x) \subset U$. Thus, 
	\begin{equation}
		(a_x, b_x) = (a_x, x] \cup [x, b_x) \subset U,
	\end{equation} 
	as desired.
\end{proof}

\subsection*{(b)}
\begin{proof}
	Let $x,y \in U$ and suppose $y\in(a_x,b_x)$. 
	
	\underline{Case 1}: $y\in (a_x,x]$. Then since $y<x$,
	\begin{align}
		a_y &= \inf\{a\in\mathbb{R} \;|\; (a, y] \subset U\} \\
		&= \inf\{a\in\mathbb{R} \;|\; (a, y]\cup[y,x] \subset U\} \\
		&= \inf\{a\in\mathbb{R} \;|\; (a, x] \subset U\}\\
		&= a_x.
	\end{align}
	Similarly,
	\begin{align}
		b_y &= \sup \{b\in\mathbb{R} \;|\; [y,b)\subset U\} \\
		&= \sup\{b\in\mathbb{R} \;|\; [x,y]\cup[y,b) \subset U\} \\
		&= \sup \{b\in\mathbb{R} \;|\; [x,b) \subset U\}\\
		&= b_x.
	\end{align}
	\underline{Case 2}: $y \in [x, b_x)$. Then since $y>x$,
	\begin{align}
		a_y &= \inf\{a\in\mathbb{R} \;|\; (a, y] \subset U\} \\
		&= \inf \{a\in\mathbb{R} \;|\; (a,x]\cup[x,y] \subset U\} \\
		&= \inf \{a\in\mathbb{R} \;|\; (a,x] \subset U\} \\
		&= a_x.
	\end{align}
	Similarly,
	\begin{align}
		b_y &= \sup\{b\in\mathbb{R} \;|\; [y, b) \subset U\} \\
		&= \sup\{b\in\mathbb{R} \;|\; [x,y]\cup[y,b) \subset U\} \\
		&= \sup\{b\in\mathbb{R} \;|\; [x,b) \subset U\} \\
		&= b_x.
	\end{align}
	So we conclude that $(a_x, b_x) = (a_y,b_y)$
\end{proof}

\subsection*{(c)}
\begin{proof}
	Let $x \in \bigcup_{q \in U\cap\mathbb{Q}}(a_q, b_q)$. 
	
	Then for at least one $q\in U\cap \mathbb{Q}$, $x\in(a_q, b_q)$. By part \textbf{(a)}, since $q\in U$, then $(a_q, b_q) \subset U \implies x \in U$. Thus,
	\begin{equation}\label{union_sub_u}
		\bigcup\limits_{q \in U\cap\mathbb{Q}} (a_q, b_q) \subseteq U.
	\end{equation}
	Now let $x \in U$. 
	
	Since $U$ is open, then $\exists \epsilon >0$ such that $(x-\epsilon, x+\epsilon) \subset U$. Since $\mathbb{Q}$ is dense in $\mathbb{R}$, then $\exists q,r \in U\cap\mathbb{Q}$ such that $x-\epsilon<q<x$ and $x<r<x+\epsilon$, i.e. 
	\begin{equation}
		q<x<r.
	\end{equation}
	By \textbf{(a)}, $(a_q, b_q)\subset U$. Then since $x>q$, we have $x\in(a_q, b_q)$ \\$\implies x \in \bigcup_{q \in U\cap\mathbb{Q}}(a_q, b_q)$. Thus,
	\begin{equation}\label{u_sub_union}
		U \subseteq \bigcup\limits_{q \in U\cap\mathbb{Q}} (a_q, b_q). 
	\end{equation}
	Then equations \eqref{union_sub_u} and \eqref{u_sub_union} imply the final result,
	\begin{equation}
		U = \bigcup\limits_{q \in U\cap\mathbb{Q}} (a_q, b_q), 
	\end{equation}
	as desired.
\end{proof}
In other words, having completed these proofs, we've shown that every open set $U\subset \mathbb{R}$ can be written as a countable union of open intervals. 

\end{document}