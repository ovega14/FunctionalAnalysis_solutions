\documentclass{article}
\usepackage[utf8]{inputenc}
\usepackage[english]{babel}
\usepackage[]{amsthm} %lets us use \begin{proof}
\usepackage[]{amssymb} %gives us the character \varnothing
\usepackage[]{amsmath}
\usepackage[parfill]{parskip} %avoid indent when skipping lines
\usepackage[toc,page]{appendix}
\usepackage{hyperref}
\hypersetup{
	colorlinks=true,
	linkcolor=blue,
	filecolor=magenta,      
	urlcolor=cyan,
	pdftitle={18.102-ps4},
	pdfpagemode=FullScreen,
}
%\urlstyle{same}

\title{18.102 Assignment 4}
\author{Octavio Vega}
\date\today

\begin{document}
\maketitle
	
\section*{Problem 1}
\begin{proof}
	Let $A \subset \mathbb{R}$ and let $E \in \mathcal{A}$. Then $f^{-1}(E)$ is measurable, so
	\begin{equation}
		m^*\left(A\cap f^{-1}(E)\right) + m^*\left(A\cap f^{-1}(E)^c\right) \leq m^*(A).
	\end{equation}
	We will first show that $\mathcal{A}$ is closed under taking complements so we must show that $E^c \in \mathcal{A}$ for every $E \in \mathcal{A}$; i.e. $f^{-1}(E^c)$ is measurable. 
	
	We will use the fact that $f^{-1}(E^c) = f^{-1}(E)^c$, which is proven in appendix \ref{appendix:A}.
	
	Since $f^{-1}(E)$ is measurable, we have
	\begin{align}
		m^*(A) &\geq m^*\left(A\cap f^{-1}(E)\right) + m^*\left(A\cap f^{-1}(E)^c\right) \\
		&= m^*\left[A\cap\left(f^{-1}(E)^c\right)^c\right] + m^*\left(A\cap f^{-1}(E^c)\right) \\
		&= m^*\left(A\cap f^{-1}(E^c)^c\right) = m^*\left(A\cap f^{-1}(E^c)\right).
	\end{align}
	Hence, $f^{-1}(E^c)$ is measurable, so $E^c \in \mathcal{A}$. Thus, $\mathcal{A}$ is closed under taking complements.
	
	Now we show that $\mathcal{A}$ is closed under taking countable unions. 
	
	Let $\{E_n\}_n \subset \mathcal{A}$ be a sequence of sets in $\mathcal{A}$, and let $A\subset \mathbb{R}$. Then 
	\begin{align}
		m^*\left[A\cap f^{-1}\left(\bigcup_n E_n\right)\right] &= m^*\left[A\cap\left(\bigcup_n f^{-1}(E_n)\right)\right] \\
		&= m^*\left[\bigcup_n\left(A\cap f^{-1}(E_n)\right)\right] \\
		& \leq \sum_n m^*\left(A\cap f^{-1}(E_n)\right) \\
		& \leq m^*\left(A\cap f^{-1}(E_n)\right),
	\end{align}
	by countable subadditivity and positive-definiteness of the outer measure $m^*$. Similarly, we have
	\begin{align}
		m^*\left[A\cap f^{-1}\left(\bigcup_n E_n\right)^c\right] &= m^*\left[A\cap \left(\bigcup_n f^{-1}(E_n)\right)^c\right] \\
		&= m^*\left[A\cap\left(\bigcap_n f^{-1}(E_n)^c\right)\right] \\
		&\leq m^*\left(A\cap f^{-1}(E_n)^c\right),
	\end{align}
	by monotonicity of $m^*$, because $\bigcap_n f^{-1}(E_n)^c \subseteq f^{-1}(E_n)^c$.
	
	Finally, since $E_n \in \mathcal{A}$, then $f^{-1}(E_n)$ is Lebesgue measurable, so we have 
	\begin{align}
		m^*\left[A\cap f^{-1}\left(\bigcup_n E_n\right)\right] + \\ m^*\left[A\cap f^{-1}\left(\bigcup_n E_n\right)^c\right]
		&\leq m^*\left(A\cap f^{-1}(E_n)\right) + m^*(A\cap f^{-1}(E_n)^c) \\
		&\leq m^*(A).
	\end{align}
	So, $f^{-1}\left(\bigcup_n E_n\right)$ is measurable, hence $\bigcup_n E_n \in \mathcal{A}$. Thus, $\mathcal{A}$ is closed under taking countable unions.
	
	Therefore, $\mathcal{A}$ is a $\sigma$-algebra.
\end{proof}
%%%%%%%%%%%%%%%%%%%%%%%%%%%%%%%%%%%%%%%%%%%%%%%%%%%%%%%%%%%%%%%%%%%%%%%%%%%%%%%%%%%%%%%%%%%%%%%%%%%%%%%%%%%%%%%%%%%%%%%%%%%%%%%%%%%%%%%%%%%%
\section*{Problem 2}
TODO TODO TODO 

%%%%%%%%%%%%%%%%%%%%%%%%%%%%%%%%%%%%%%%%%%%%%%%%%%%%%%%%%%%%%%%%%%%%%%%%%%%%%%%%%%%%%%%%%%%%%%%%%%%%%%%%%%%%%%%%%%%%%%%%%%%%%%%%%%%%%%%%%%%%
\begin{appendices}
	
\section{Appendix A}
\label{appendix:A}
\textbf{Theorem:} Let $f:\mathbb{R}\rightarrow\mathbb{R}$. If $E \subset \mathbb{R}$, then $f^{-1}(E^c) = f^{-1}(E)^c$.
\begin{proof}
	Let $x \in f^{-1}(E^c)$. Then
	\begin{align}
		&\implies f(x) \in E^c \iff f(x) \notin E \\
		&\implies x \notin f^{-1}(E) \iff x \in f^{-1}(E)^c.
	\end{align}
	Thus, we have
	\begin{equation}\label{subset1}
		f^{-1}(E^c) \subseteq f^{-1}(E)^c.
	\end{equation}
	Now let $x\in f^{-1}(E)^c$. Then
	\begin{align}
		&\implies x \notin f^{-1}(E) \iff f(x) \notin E \\
		&\implies f(x) \in E^c \iff x \in f^{-1}(E^c).
	\end{align}
	So, we also have
	\begin{equation}\label{subset2}
		f^{-1}(E)^c \subseteq f^{-1}(E^c).
	\end{equation}
	Taking equations \eqref{subset1} and \eqref{subset2} together allows us to conclude that \\$f^{-1}(E^c) = f^{-1}(E)^c$, as desired.
\end{proof}
%%%%%%%%%%%%%%%%%%%%%%%%%%%%%%%%%%%%%%%%%%%%%%%%%%%%%%%%%%%%%%%%%%%%%%%%%%%%%%%%%%%%%%%%%%%%%%%%%%%%%%%%%%%%%%%%%%%%%%%%%%%%%%%%%%%%%%%%%%%%
\section{Appendix B}
\label{appendix:B}

\textbf{Theorem:} Let $f:\mathbb{R} \rightarrow \mathbb{R}$ and let $\{E_n\}_n$ be a collection of subsets $E_n \subset \mathbb{R}$. Then
\begin{equation}
	f^{-1}\left(\bigcup_n E_n\right) = \bigcup_n f^{-1}(E_n).
\end{equation}
\begin{proof}
	Let $x \in f^{-1}\left(\cup_n E_n\right)$. Then $f(x) \in \cup_n E_n$, so $f(x)$ is in any of $E_n$ for $n\in\mathbb{N}$. This is equivalent to saying that $x \in f^{-1}(E_n)$ for some $n \in \mathbb{N}$, which implies that $x \in \cup_n f^{-1}(E_n)$. Hence,
	\begin{equation}
		f^{-1}\left(\bigcup_n E_n\right) \subseteq \bigcup_n f^{-1}(E_n).
	\end{equation}
	Now let $x \in \cup_n f^{-1}(E_n)$. Then $x \in f^{-1}(E_n)$ for some $n \in \mathbb{N}$, which is equivalent to saying that $f(x) \in E_n$ for some $n$. Then $f(x) \in \cup_n E_n$, so $x \in f^{-1}\left(\cup_n E_n\right)$. Thus, 
	\begin{equation}
		\bigcup_n f^{-1}(E_n) \subseteq f^{-1}\left(\bigcup_n E_n\right).
	\end{equation}
	Therefore, both sets are subsets of one another, so they are equal.
\end{proof}

\end{appendices}
	
\end{document}